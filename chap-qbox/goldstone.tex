计算 Goldstone 图是 $\hat{Q}$-box 计算最重要的部分。$\hat{Q}$-box 最开始是一系列价空间外组态的求和,理论上说对组态进行循环计算就能得到结果,并不需要引入 Goldstone 图。但由于组态的数量过多,引入 Goldstone 图的目的是把对组态的求和转为对单粒子轨道的求和,实际做的事情是把包含组态之间的矩阵元的公式用图及其计算规则,转换为了用 Fock 空间矩阵元表达的形式。在此基础上,由于 $m$-scheme 的轨道仍然很多,在 $j$-scheme 下进行单粒子轨道求和才是可行的,这就需要一系列的角动量耦合,把 $m$-scheme 的 Goldstone 图在耦合表象下写出。

Goldstone 图的归一化问题写一下。

接下来的推导遵循 \cite{jin-goldstone} 的定义。需要指出的是,之前定义过单粒子态的量子数标记为 $|\zeta_aj_am_a\rangle$ 或 $|k_am_a\rangle$,然而在涉及 Goldstone 图的计算中,写太多下标很复杂,因此在波函数中用 $a$ 表示 $(\zeta_a,j_a)$,分别用 $\alpha,\beta,\gamma,\delta,...$ 表示 $m_a,m_b,m_c,m_d,...$;而在 CG 系数,$3j$ 符号,相因子等数学结构中,用 $a$ 表示 $j_a$ 这个数。

\section{基本定义}
耦合表象的相互作用矩阵元为
\begin{equation}
  V_{abcd}^J=\sum_{\alpha\beta\gamma\delta}C_{a\alpha b\beta}^{JM}C_{c\gamma d\delta}^{JM}\langle a\alpha b\beta|\hat{V}|c\gamma d\delta\rangle,\label{eqs:goldstone-V}
\end{equation}
定义两种交叉耦合矩阵元,角动量耦合的方向与 $V_{abcd}^J$ 不同,
\begin{align}
  R_{abcd}^J&=\sum_{\alpha\beta\gamma\delta}(-1)^{b-\beta}C_{b-\beta a\alpha}^{JM}(-1)^{d-\delta}C_{d-\delta c\gamma}^{JM}\langle a\alpha d\delta|\hat{V}|b\beta c\gamma\rangle,\label{eqs:goldstone-R}\\
  T_{abcd}^J&=\sum_{\alpha\beta\gamma\delta}(-1)^{b-\beta}C_{a\alpha b-\beta}^{JM}(-1)^{d-\delta}C_{c\gamma d-\delta}^{JM}\langle a\alpha d\delta|\hat{V}|c\gamma b\beta\rangle.\label{eqs:goldstone-T}
\end{align}

\section{改变耦合方向}
为了更好地使用新定义的三种耦合矩阵元及其分解规则,对于复杂的图需要将耦合方向改变,才能将图用这些分解规则拆开。改变耦合方向本质是 $V,R,T$ 三类矩阵元之间的关系,推导如下。

\textbullet $V-R$ 关系

首先写出 $V$ 的定义 \ref{eqs:goldstone-V},需要在右侧插入两个 CG 系数的幺正性 \ref{app:cg-unitary-sumJM},但由于在对 $\alpha\beta\gamma\delta$ 求和之后还必须剩下一组对 $\alpha\beta\gamma\delta$ 的求和,才能最终凑出 $R$ 矩阵元,因此需要先把 \ref{eqs:goldstone-V} 改写为两次对 $\alpha\beta\gamma\delta$ 求和的形式,
\begin{equation}
  V_{abcd}^J=\sum_{\alpha'\beta'\gamma'\delta'}\sum_{\alpha\beta\gamma\delta}C_{a\alpha b\beta}^{JM}C_{c\gamma d\delta}^{JM}\langle a\alpha' b\beta'|\hat{V}|c\gamma' d\delta'\rangle\delta_{\alpha\alpha'}\delta_{\beta\beta'}\delta_{\gamma\gamma'}\delta_{\delta\delta'},
\end{equation}
因此可以把右侧的 $\delta$ 换为 CG 系数的幺正关系 \ref{app:cg-unitary-sumJM}, 
\begin{equation}
  \sum_{J'M'}C_{b\beta c-\gamma}^{J'M'}C_{b\beta'c-\gamma'}^{J'M'}=\delta_{\beta\beta'}\delta_{\gamma\gamma'},\quad \sum_{J''M''}C_{a-\alpha d\delta}^{J''M''}C_{a-\alpha' d\delta'}^{J''M''}=\delta_{\alpha\alpha'}\delta_{\delta\delta'},
\end{equation}
因此右侧的 CG 系数部分变为 
\begin{align}
  &\sum_{\alpha'\beta'\gamma'\delta'}\sum_{\alpha\beta\gamma\delta}\sum_{J'M'J''M''}C_{a\alpha b\beta}^{JM}C_{c\gamma d\delta}^{JM}C_{b\beta c-\gamma}^{J'M'}C_{b\beta' c-\gamma'}^{J'M'}C_{a-\alpha d\delta}^{J''M''}C_{a-\alpha' d\delta'}^{J''M''}\notag\\
  ={}&\sum_{\alpha'\beta'\gamma'\delta'}\sum_{J'M'J''M''}C_{b\beta' c-\gamma'}^{J'M'}C_{a-\alpha' d\delta'}^{J''M''}\left(\sum_{\alpha\beta\gamma\delta}C_{a\alpha b\beta}^{JM}C_{c\gamma d\delta}^{JM}C_{b\beta c-\gamma}^{J'M'}C_{a-\alpha d\delta}^{J''M''}\right),
\end{align}
括号里的四个 CG 系数转换为 $6j$ 符号,留下的两个 CG 系数以及四个对 $m$ 量子数的求和与 $m$-scheme 矩阵元组合成 $R$,所以需要按照上式的方式分配。然而这里还存在一个问题,括号里的 CG 系数 $C_{a\alpha b\beta}^{JM}$ 与 $C_{c\gamma d\delta}^{JM}$ 需要用 \ref{app:cg-changej1j2j3} 对三个角动量进行翻转,才能凑成正确的 \ref{app:cg-to-6j} 的形式。然而翻转会产生相位 $(-1)^{a-\alpha}$ 与 $(-1)^{c-\gamma}$,这两个相位的 $\alpha,\gamma$ 量子数被求和了,这非常危险,这些相位在 \ref{app:cg-to-6j} 中是不该出现的。因此我们在最一开始就把 $V_{abcd}^J$ 中出现的这两个 CG 系数翻转为需要的顺序,
\begin{equation}
    V_{abcd}^J=\sum_{\alpha'\beta'\gamma'\delta'}\sum_{\alpha\beta\gamma\delta}(-1)^{a+J-b}(-1)^{a-\alpha'}\frac{\hat{J}}{\hat{b}}C_{a-\alpha JM}^{b\beta}(-1)^{c-\gamma'}\frac{\hat{J}}{\hat{d}}C_{JM c-\gamma}^{d\delta}
    \langle a\alpha' b\beta'|\hat{V}|c\gamma' d\delta'\rangle\delta_{\alpha\alpha'}\delta_{\beta\beta'}\delta_{\gamma\gamma'}\delta_{\delta\delta'},
\end{equation}
在把 $\delta$ 换为 CG 系数的幺正性之前,将 $\alpha,\gamma$ 替换为 $\alpha',\gamma'$ 是安全的。这样就能安全地使用 \ref{app:cg-to-6j},技巧是分析两种耦合顺序,需要最后剩下 $\delta_{J'J''}$ 才能消去其中一个 $J$ 求和,因此一种耦合顺序是 $a$ 与 $J$ 耦合为 $b$,$b$ 与 $c$ 再耦合为 $J'$;另一种耦合顺序是 $J$ 与 $c$ 耦合为 $d$,$a$ 与 $d$ 再耦合为 $J''$。因此得到
\begin{align}
  &\sum_{J'M'J''M''}C_{b\beta' c-\gamma'}^{J'M'}C_{a-\alpha' d\delta'}^{J''M''}\sum_{\alpha\beta\gamma\delta}C_{a-\alpha JM}^{b\beta}C_{JM c-\gamma}^{d\delta}C_{b\beta c-\gamma}^{J'M'}C_{a-\alpha d\delta}^{J''M''}\notag\\
  ={}&\sum_{J'M'J''M''}C_{b\beta' c-\gamma'}^{J'M'}C_{a-\alpha' d\delta'}^{J''M''}\sum_{\alpha\beta\gamma\delta}C_{a-\alpha JM}^{b\beta}C_{b\beta c-\gamma}^{J'M'}C_{JM c-\gamma}^{d\delta}C_{a-\alpha d\delta}^{J''M''}\notag\\
  ={}&\frac1{\hat{J}^2}\sum_{J'J''M'M''}C_{b\beta' c-\gamma'}^{J'M'}C_{a-\alpha' d\delta'}^{J''M''}\sum_{\alpha\beta\gamma\delta M}C_{a-\alpha JM}^{b\beta}C_{b\beta c-\gamma}^{J'M'}C_{JM c-\gamma}^{d\delta}C_{a-\alpha d\delta}^{J''M''}\notag\\
  ={}&\frac1{\hat{J}^2}\sum_{J'J''M'M''}C_{b\beta' c-\gamma'}^{J'M'}C_{a-\alpha' d\delta'}^{J''M''}\delta_{J'J''}\delta_{M'M''}(-1)^{a+c+J+J'}\hat{b}\hat{d}
  \begin{Bmatrix}
    a&b&J\\c&d&J'
  \end{Bmatrix}\notag\\
  ={}&\frac1{\hat{J}^2}\sum_{J'}\hat{J'}^2C_{b\beta' c-\gamma'}^{J'M'}C_{a-\alpha' d\delta'}^{J'M'}(-1)^{a+c+J+J'}\hat{b}\hat{d}
  \begin{Bmatrix}
    a&b&J\\c&d&J'
  \end{Bmatrix},
\end{align}
代入 $V_{abcd}^J$ 中,得到 
\begin{align}
  V_{abcd}^J&=\sum_{J'}
  \begin{Bmatrix}
    a&b&J\\c&d&J'
  \end{Bmatrix}
  \hat{J'}^2\sum_{\alpha'\beta'\gamma'\delta'}(-1)^{a-\alpha'}C_{a-\alpha' d\delta'}^{J'M'}(-1)^{c-\gamma'}C_{c-\gamma'b\beta'}^{J'M'}\langle a\alpha' b\beta'|\hat{V}|c\gamma' d\delta'\rangle\notag\\
  &=-\sum_{J'}
  \begin{Bmatrix}
    a&b&J\\c&d&J'
  \end{Bmatrix}\hat{J'}^2
  \sum_{\alpha'\beta'\gamma'\delta'}(-1)^{a-\alpha'}C_{a-\alpha' d\delta'}^{J'M'}(-1)^{c-\gamma'}C_{c-\gamma'b\beta'}^{J'M'}\langle d\delta'c\gamma'|\hat{V}|a\alpha'b\beta'\rangle\notag\\
  &=-\sum_{J'}
  \begin{Bmatrix}
    a&b&J\\c&d&J'
  \end{Bmatrix}\hat{J'}^2R_{dabc}^{J'}.
\end{align}
