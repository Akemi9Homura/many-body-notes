前一章讨论的是应用于简并模型空间的 KK 与 LS 方法。而用到两个主壳层的 HO-MBPT 以及 HF-MBPT 要求在非简并模型空间中求解有效哈密顿量。将 KK 与 LS 推广到非简并模型空间文献中有两种做法,一种是通过多能 $\hat{Q}$-box 得到 generalized KK 或 LS 方法,记为 GKK 与 GLS;另一种推广则称为 extended KK 或 LS 方法,记为 EKK 与 ELS,与简并模型空间中的 KK,LS 相比,每步迭代中更新的是 $H^\text{eff}$ 而不是 $H_1^\text{eff}$。另外,\cite{2014dong} 提到在 HO 基下如果跨壳可能会有质心运动的问题,这需要考虑一下。

简并模型空间的方法不能直接应用于非简并模型空间,首先理论上说不通,另一方面是可能导致 $\hat{Q}$-box 不收敛。
\begin{figure}[htbp]
  \centering
  \includegraphics[width=0.3\textwidth]{figure/corepol.png}
  \caption{两体二阶核心极化图。图取自 \cite{2014tsunoda}。}
  \label{fig:corepol}
\end{figure}
以图 \ref{fig:corepol} 为例,EKK 中的能量分母为 $(E-\epsilon_c-\epsilon_b-\epsilon_p+\epsilon_h)$,而 KK 则是 $[(\epsilon_c+\epsilon_d)-\epsilon_c-\epsilon_b-\epsilon_p+\epsilon_h]=\epsilon_d-\epsilon_b-\epsilon_p+\epsilon_h$,如果在 HO-MBPT 中取模型空间为 $sdpf$ 壳,当 $a,d\in 1p0f$,$b,c,p\in 1s0d$ 以及 $h\in 0p$ 时,$\epsilon_d=3\hbar\omega,\epsilon_b=\epsilon_p=2\hbar\omega,\epsilon_h=\hbar\omega$,能量分母为 0,KK 方法无法使用。

\section{EKK}
EKK 的迭代公式为
\begin{align}
  &QH_1P+QHQ\omega_n-\omega_nPHP-\omega_n PH_1Q\omega_{n-1}=0,\label{eqs:ekk-decouple}\\
  &H_n^\text{eff}=PHP+PH_1Q\omega_n.\label{eqs:nondeg-iter-heff}
\end{align}
EKK 的收敛性
\begin{equation}
  \rho_P=\frac{\langle\Psi_i|P|\Psi_i\rangle}{\langle\Psi_i|\Psi_i\rangle}>\frac12,\quad i=1,\ldots,d,
\end{equation}

在 \cite{2011takayanagi2} 中将 EKK 迭代过程分为两种方案:一种是将 $H_{n}^\text{eff}$ 展开为 $H_{n-1}^\text{eff}$ 的级数,称为 EKKS;另一种则是将 $H_{n-1}^\text{eff}$ 对角化,称为 EKKD。

\subsection{EKKD——对角化的 EKK}
将 \ref{eqs:nondeg-iter-heff} 代入 \ref{eqs:ekk-decouple} 中,得到
\begin{equation}
  \omega_nH_{n-1}^\text{eff}-QHQ\omega_n=QH_1P,\label{eqs:nondeg-iter-ekkd}
\end{equation}
假设已经得到了 $H_{n-1}^{\mathrm{eff}}$,在 $P$ 空间进行对角化,
\begin{equation}
  H_{n-1}^{\mathrm{eff}}|\phi_i^{(n-1)}\rangle=E_i^{(n-1)}|\phi_i^{(n-1)}\rangle,\quad i=1,\ldots,d,\label{eqs:n-1 diagonalization}
\end{equation}
得到一组本征值 $E_i^{(n-1)}$ 与本征态 $|\phi_i^{(n-1)}\rangle$,将 $|\phi_i^{(n-1)}\rangle$ 右乘到 \ref{eqs:nondeg-iter-ekkd} 上,可得
\begin{equation}
  \omega_n|\phi_i^{(n-1)}\rangle=\frac1{E_i^{(n-1)}-QHQ}QH_1P|\phi_i^{(n-1)}\rangle,\quad i=1,\ldots,d,
\end{equation}
这就得到了 $\omega_n$,代入 \ref{eqs:nondeg-iter-heff} 中得到第 $n$ 步的 $H_{n}^{\mathrm{eff}}$,
\begin{align}
  H_{n}^{\mathrm{eff}}|\phi_i^{(n-1)}\rangle&=\left(PHP+PH_1Q\frac1{E_i^{(n-1)}-QHQ}QH_1P\right)|\phi_i^{(n-1)}\rangle\notag\\
  &=H^\text{BH}(E_i^{(n-1)})|\phi_i^{(n-1)}\rangle,
\end{align}
\cite{2011takayanagi1} 给出了另一种推导,不是将 $|\phi_i^{(n-1)}\rangle$ 右乘到 \ref{eqs:nondeg-iter-ekkd} 上,而是右乘到 EKKS 的迭代公式 \ref{eqs:nondeg-iter-ekks} 上,等式右边就会变成 $\hat{Q}(E_i^{(n-1)})$ 以 $E$ 为起点的泰勒展开,
\begin{equation}
  H_{n}^{\mathrm{eff}}|\phi_i^{(n-1)}\rangle=PH_0P|\phi_i^{(n-1)}\rangle+\sum_{k=0}^\infty\hat{Q}(E)(E_i^{(n-1)}-E)^k=H^\text{BH}(E_i^{(n-1)})|\phi_i^{(n-1)}\rangle,\label{eqs:ekks-taylor}
\end{equation}
也就是说如果 EKKD 与 EKKS 迭代起点相同,给出的有效哈密顿量序列就是相同的 \cite{2011takayanagi2}。

由于 $|\phi_i^{(n-1)}\rangle$ 不一定是 $P$ 空间的正交基,需要构造一组正交基 $|\widetilde{\phi}_i^{(n-1)}\rangle$ 满足
\begin{equation}
  \langle\widetilde{\phi}_i^{(n-1)}|\phi_j^{(n-1)}\rangle=\delta_{ij},\quad P=\sum_{i=1}^d |\phi_i^{(n-1)}\rangle\langle\widetilde{\phi}_i^{(n-1)}|,\label{eqs:ekkd-norm}
\end{equation}
因此第 $n$ 步的有效哈密顿量可构造为
\begin{equation}
  H_{n}^{\mathrm{eff}}=\sum_{i=1}^d H^\text{BH}(E_i^{(n-1)})|\phi_i^{(n-1)}\rangle\langle\widetilde{\phi}_i^{(n-1)}|.\label{eqs:n-step-heff}
\end{equation}
注意在实际计算中正交基的构建是通过矩阵求逆获得的。将 $H_{n-1}^{\mathrm{eff}}$ 对角化,获得 $d$ 个本征值与本征向量为列组成的矩阵 $\Phi=[\phi^{(n-1)}_1,\ldots,\phi^{(n-1)}_d]$,将矩阵取逆,得到
\begin{equation}
  \Phi^{-1}=
  \begin{bmatrix}
    \widetilde{\phi}_1^{(n-1)T}\\
    \vdots\\
    \widetilde{\phi}_d^{(n-1)T},
  \end{bmatrix}
\end{equation}
根据逆矩阵的定义就可以导出正交归一关系 \ref{eqs:ekkd-norm}。

迭代的起点需要注意。根据上面给出的迭代过程,需要最初选取一个哈密顿量对角化,再用这组本征值与本征向量构建下一步对角化的哈密顿量。\cite{2011takayanagi2} 的初值选取是 $PHP=PH_0P+PH_1P$,这应该是个很合理的选择,直接选取原始哈密顿量的 $P$ 空间块。而且在 EKKD 中没给初始的 $E$,也没法与 EKKS 取相同的初值。在 EKKS 代码中,将迭代的初值选为 $H^\text{eff}_0=PH_0P+\hat{Q}(E)$,在后面有效算符的计算中会发现这么取是合理的,与 KK 方法取 $H_1^\text{eff}(\omega_0)=\hat{Q}(\epsilon_0)$ 也是对应的,不过与 EKKD 的初值选取不同,就得不到与 EKKD 一致的收敛序列了。

对于价空间轨道 $a$ 的 $\hat{S}$-box,
\begin{equation}
  \langle a|PHP|a\rangle=\epsilon_a+U_{aa},
\end{equation}
对于 $\hat{Q}$-box 的 channel $J^P$ 及其中的两个两体组态 $|ab\rangle,|cd\rangle$,
\begin{equation}
  \langle ab|PHP|cd\rangle=(\epsilon_a+\epsilon_b+U_{aa}+U_{bb})\delta_{ac}\delta_{bd}+V_{abcd}^J,
\end{equation}
这里的第一项是对角部分,只有 $|ab\rangle=|cd\rangle$ 才会有,由于代码中 $|ab\rangle$ 组态 $ab$ 的排序是确定的,因此 $|ab\rangle=|cd\rangle$ 对应的就是 $a=c,b=d$。这一项有两种理解方式,一种是 $PHP$ 的单体部分作用到两体空间,也就是后面证明的 \ref{eqs:real-1b-to-2b};另一种则是发现 $PH_1P$ 就是 $\hat{Q}$-box 的一阶图。

EKKD 直接用了 EKK 最原始的迭代式 \ref{eqs:ekk-decouple} 与 \ref{eqs:nondeg-iter-heff} 进行推导,因此 EKKD 的收敛条件就是 

相比 EKKD,EKKS 在推导中 \ref{eqs:ekks-omega} 进行了级数展开。 

\subsection{EKKS——级数求和的 EKK}
在 \ref{eqs:nondeg-iter-ekkd} 中的 $H_{n-1}^\text{eff}$ 减去一个任意参数 $E$,
\begin{equation}
  (E-QHQ)\omega_n=QH_1P-\omega_n(H_{n-1}^\text{eff}-E),\label{eqs:ekks-decouple}
\end{equation}
由此可形式地解出 $\omega_n$,等式两侧左乘 $(E-QHQ)^{-1}$ 并进行级数展开,得到
\begin{equation}
  \omega_n=\frac{QH_1P}{E-QHQ}\left(1+\frac{H_{n-1}^\text{eff}-E}{E-QHQ}\right)^{-1}=\sum_{k=0}^\infty(-1)^k\frac{1}{(E-QHQ)^{k+1}}QH_1P(H_{n-1}^\text{eff}-E)^k,\label{eqs:ekks-omega}
\end{equation}
再代入 \ref{eqs:nondeg-iter-heff} 得到
\begin{align}
  H_n^\text{eff}&=PH_0P+\sum_{k=0}^\infty\hat{Q}_k(E)(H_{n-1}^\text{eff}-E)^k\notag\\
  &=H^\text{BH}(E)+\sum_{k=1}^\infty\hat{Q}_k(E)(H_{n-1}^\text{eff}-E)^k,\label{eqs:nondeg-iter-ekks}
\end{align}
其中 $H^\text{BH}(E)=PH_0P+\hat{Q}(E)$。$\hat{Q}(\epsilon)$ 中 $(\epsilon-QHQ)^{-1}$ 直接计算是不可能的,需要进行微扰展开,
\begin{align}
  \hat{Q}(\epsilon)={}&P H_1 P + P H_1 Q \frac{1}{\epsilon - Q H_0 Q} Q H_1 P \notag\\
  &+ P H_1 Q \frac{1}{\epsilon - Q H_0 Q} Q H_1 Q \frac{1}{\epsilon - Q H_0 Q} Q H_1 P + \dots,
\end{align}
展开的一、二、三项分别有 1,2,3 个 $H_1$,对应一、二、三阶图。可以证明如果 EKKD 与 EKKS 迭代起点相同,给出的有效哈密顿量序列就是相同的 \cite{2011takayanagi2}。那么应该如何理解 EKKD 没有 $E$ 这个参数,而 EKKS 却有呢?$E$ 可以看作是 $H^\text{BH}(E_i^{(n-1)})$ 泰勒展开的起点,如果泰勒级数能够收敛,其结果与起点的选取无关。需要注意的是 \ref{eqs:nondeg-iter-ekks} 并不是泰勒展开,这只是一个有效哈密顿量满足的自洽迭代公式,泰勒展开出现在 \ref{eqs:ekks-taylor} 中。

在 \cite{2011takayanagi1} 的推导中,没有减去起点能量 $E$,而是直接写出
\begin{equation}
  \omega_n=\sum_{k=0}^\infty(-1)^k\frac{1}{(-QHQ)^{k+1}}QH_1P(H_{n-1}^\text{eff})^k,
\end{equation}
相当于计算了 $\epsilon=0$ 的 $\hat{Q}(\epsilon)$,因此 
\begin{align}
  H_n^\text{eff}&=PH_0P+\sum_{k=0}^\infty\hat{Q}_k(0)(H_{n-1}^\text{eff})^k\notag\\
  &=H^\text{BH}(0)+\sum_{k=1}^\infty\hat{Q}_k(0)(H_{n-1}^\text{eff})^k,
\end{align}
\cite{2011takayanagi1} 指出,这个有效哈密顿量对泰勒展开的起点 $E=0$ 没有强烈的依赖,对此的解释是在 \ref{eqs:n-1 diagonalization} 的求解中 $E_i^{(n-1)}$ 会吸收起点的改变,经过几步迭代之后起点的影响就看不出来了。不过这个解释有些奇怪,EKKS 并不会进行 EKKD 的对角化这一步骤,这是否意味着可以在 EKKS 之后跟上几步 EKKD,以消除对起点的依赖?

这之前的推导都是形式地将 $H$ 分为 $H_0$ 与 $H_1$,并没有用到 $H_0$ 与 $H_1$ 的具体形式,因此可以进行选择。输入的二次量子化形式的哈密顿量为
\begin{equation}
  H=\sum_{\alpha\beta}h_{\alpha\beta}a^\dagger_\alpha a_\beta+\frac14\sum_{\alpha\beta\gamma\delta}V_{\alpha\beta\gamma\delta}a^\dagger_\alpha a^\dagger_\beta a_\delta a_\gamma=H_0+H_1,
\end{equation}
在 HF 基下,自然的选取为
\begin{align}
  H_0&=\sum_\alpha\epsilon_\alpha a^\dagger_\alpha a_\alpha,\\
  H_1&=\sum_{\alpha\beta}(h_{\alpha\beta}-\epsilon_\alpha\delta_{\alpha\beta})a^\dagger_\alpha a_\beta+\frac14\sum_{\alpha\beta\gamma\delta}V_{\alpha\beta\gamma\delta}a^\dagger_\alpha a^\dagger_\beta a_\delta a_\gamma\notag\\
  &=\sum_{\alpha\beta}g_{\alpha\beta}a^\dagger_\alpha a_\beta+\frac14\sum_{\alpha\beta\gamma\delta}V_{\alpha\beta\gamma\delta}a^\dagger_\alpha a^\dagger_\beta a_\delta a_\gamma,
\end{align}
EKK 可以看作将 $H_0$ 的价空间部分强行简并为 $E$,再使用 KK 的公式,也就是说 
\begin{equation}
  H=H_0'+H_1'
  =\begin{pmatrix}
    E&0\\0&QH_0Q
  \end{pmatrix}
  +\begin{pmatrix}
    P\widetilde{H}P&PH_1Q\\QH_1P&QH_1Q
  \end{pmatrix},
\end{equation}
其中 $\widetilde{H}=H-E$,因此从以上划分可以直接将 KK 的简并能量 $\epsilon_0$ 换为 $E$,被迭代的 $H_1^\text{eff}$ 换为 $\widetilde{H}^\text{eff}$ 即可。但需注意首项 $\hat{Q}(E)$ 变为了
\begin{equation}
  P\widetilde{H}P+PH_1Q\frac1{E-QHQ}QH_1P=PH_0P+\hat{Q}(E)-E\equiv\widetilde{H}^\text{BH}(E).
\end{equation}

实际计算中发现 $(V-U)$ 项是比较大的,而 MBPT 所谓的微扰就是体现在

\section{ELS}
不同的求解方法处理的都是 \ref{eqs:origin-decouple} 这个解耦条件,实际迭代时需要用 $n$ 步与 $n-1$ 步 $\omega$ 的关系进行计算。EKK 写为 \ref{eqs:ekk-decouple},而 ELS 的写法有些区别,写为 
\begin{equation}
  QH_1P+QHQ\omega_n-\omega_{n-1}PHP-\omega_{n-1} PH_1Q\omega_{n}=0,\label{eqs:els-decouple}
\end{equation}
把 \ref{eqs:nondeg-iter-heff} 代入,得到 
\begin{equation}
  \omega_n=\frac1{-QHQ}QH_1P-\frac1{-QHQ}\omega_{n-1}H_n^\text{eff},
\end{equation}
再代入到 \ref{eqs:nondeg-iter-heff} 中,
\begin{equation}
  H_n^\text{eff}=\left(1+PH_1Q\frac{1}{-QHQ}\omega_{n-1}\right)^{-1}H^\text{BH}(0).
\end{equation}
与 LS 的推导类似,取 $\omega_0=0$,有
\begin{align}
  & H_{1}^{\mathrm{eff}}=H^{\mathrm{BH}}(0), \\
  & H_{2}^{\mathrm{eff}}=\frac{1}{1-\hat{Q}_1(0)}H^{\mathrm{BH}}(0), \\
  & H_{3}^{\mathrm{eff}}=\frac{1}{1-\hat{Q}_1(0)-\hat{Q}_2(0)H_2^{\mathrm{eff}}}H^{\mathrm{BH}}(0),
\end{align}
因此得到
\begin{equation}
  H_n^\text{eff}=\left(1-\hat{Q}_1(0)-\sum_{m=2}^{n-1}\hat{Q}_m(0)\prod_{k=n-m+1}^{n-1}H_k^\text{eff}\right)^{-1}H^{\mathrm{BH}}(0).
\end{equation}
\cite{2011takayanagi1} 引入的 EKK 与 ELS 方法相比简并模型空间的 KK 与 LS 方法,可以直接把 $\epsilon_0$ 换为 0 或一个任意的起点能量 $E$,把 $H_1^\text{eff}(\omega_n)$ 替换为 $H_n^\text{eff}$,并且在等式两侧同时加上 $PH_0P$,此时就可以把 $PH_0P+\hat{Q}(E)$ 组合为 $H^\text{BH}(E)$。不过注意这里的 EKK 指的是 EKKS,因为 \cite{2012coraggio} 中引入的简并模型空间的 KK 就是级数求和形式的,对应的非简并模型空间的推广就也是级数求和形式的 EKKS。但正如前所述 \cite{2011takayanagi2} 与 \cite{2014dong} 还给出了 EKKD,这也是一种 KK 向非简并模型空间的推广。

$\hat{Q}$-box 都是计算能量为 0 处肯定很不好,需要引入参数 $E$。这里有两种方法,其一是遵循 \cite{2011takayanagi1,2011takayanagi2} 的一贯思路,这两篇文章推导 EKK 从迭代式 \ref{eqs:ekk-decouple} 出发的,推导 ELS 则是从迭代式 \ref{eqs:els-decouple},也就是从最初就在解耦条件中引入下标 $n$ 与 $(n-1)$。对于 EKKS 能方便地引入 $E$,因为 \ref{eqs:ekk-decouple} 只有$\omega_n$,在 $QHQ$ 上减去再加上 $E$ 就能凑出想要的形式。而 \ref{eqs:els-decouple} 则不能这么推导,只能先推导出一个 $E=0$ 的形式。在此基础上,对 $\widetilde{H}=H-E$ 写出
\begin{equation}
  \widetilde{H}_n^\text{eff}=\left(1-\hat{Q}'_1(0)-\sum_{m=2}^{n-1}\hat{Q}'_m(0)\prod_{k=n-m+1}^{n-1}\widetilde{H}_k^\text{eff}\right)^{-1}\widetilde{H}^{\mathrm{BH}}(0),
\end{equation}
认为 $E$ 的平移是作用到 $H_0$ 中的,$H_1$ 不变,因此有 $\hat{Q}'_k(0)=\hat{Q}_k(E),\widetilde{H}^{\mathrm{BH}}(0)=H^{\mathrm{BH}}(E)-E$,这样就得出
\begin{equation}
  \widetilde{H}_n^\text{eff}=\left(1-\hat{Q}_1(E)-\sum_{m=2}^{n-1}\hat{Q}_m(E)\prod_{k=n-m+1}^{n-1}\widetilde{H}_k^\text{eff}\right)^{-1}\left(H^{\mathrm{BH}}(E)-E\right).
\end{equation}
另一种推导方法是先写出自洽方程,也就是等式不含有步数的下标,然后再改写为迭代式的形式。之前 KK 与 LS 的推导都是这么做的。得到的结果为
\begin{align}
  \omega&=\frac1{E-QHQ}QH_1P-\frac1{E-QHQ}\omega\widetilde{H}^\text{eff},\\
  \widetilde{H}^\text{eff}&=\left(1+PH_1Q\frac{1}{E-QHQ}\omega\right)^{-1}\left(H^\text{BH}(E)-E\right).
\end{align}

此外 \cite{2011takayanagi1} 还指出与 EKKS 不同,起点能量 $E=0$ 对 ELS 很重要

\section{EKK 与 ELS 总结}
% \subsection{EKK 与 ELS 的比较}
% 可以证明 EKK 得到的有效哈密顿量的本征态是与 $P$ 空间有最大重叠的态,而 ELS 得到的有效哈密顿量的本征值是完整的

\subsection{EKK 起点能量 $E$ 的选取}
在 EKK 与 ELS 中,$E$ 的选取要达到两个目的,一是需要满足收敛条件 \ref{eqs:ekks-conv},二是需要保证 $\hat{Q}$-box 的每个分母都不是 0,也就是 $E$ 要避开 $\hat{Q}$-box 的极点,否则在计算过程中就会发散。

\ce{^43Sc} 以 \ce{^28Si} 为核芯的 EKKS 计算可以说明这些现象。
将 $\hat{S}$-box 的 $E_1$ 取为 $-7$ MeV,$\nu(1s_{1/2})$ 的 $\hat{S}$-box 的导数随着阶数的增大而增大,这是因为 $k$ 阶导数的分母上是 $(E_1-(\sum\epsilon_p-\sum\epsilon_h))^{k+1}$,如果 $E_1-(\sum\epsilon_p-\sum\epsilon_h)<1$,高阶导数就会很大,而要想使 EKKS 与起点能量无关,就需要求和到所有阶 \cite{2014dong}。因此 $E$ 的选择要避开 $\hat{Q}$-box 分母的极点,判断标准就是所有分母都要大于 1,这样随着求导阶数的升高才能逐渐压低。

不过即使如此也有可能不收敛。将 $E_1$ 改为 $-20$ MeV,$\hat{S}$-box 的导数不发散了,但此时价空间单粒子能仍然发散。将级数求和中相邻的两项相比,发现某些项
\begin{equation}
  \left|\frac{\hat{Q}_{k+1}(E_1)}{\hat{Q}_{k}(E_1)}(E_{\mathrm{sp}}-E_1)\right|>1,
\end{equation}
这导致了级数随着项数的增加反而变大,最终发散。将 $(V-U)$ 图去掉后同样的 $E_1$ 可以收敛,这说明 Hartree-Fock 基的非微扰单粒子能的改变并不是进行分数填充之后不收敛的原因。同时观察到 $(V-U)$ 图的加入对 $\hat{Q}(\epsilon)$ 产生了明显的改变,应该是 $\hat{Q}(\epsilon)$ 性质的变化导致对于同样的起点能量 EKKS 变得不收敛。但 $(V-U)$ 图并未直接出现在级数收敛的条件 \ref{eqs:ekks-conv} 中,这就引出了一些疑问:$(V-U)$ 图的加入对 $E_p$ 的间接改变大到了什么地步?怎么才能像 \cite{2011takayanagi2} 指出的,在计算之前就能先验地判断 $E_p$ 并将 $E$ 取到 $E_p$ 附近?根据单粒子能的实验值吗?但是在分数填充并对所计算的原子核进行 Hartree-Fock 的情境下,单粒子能实验值真的有意义吗?是针对这个要计算的原子核的吗?

以上问题暂时没有一个让人满意的解答。

\section{EKK 的收敛性}
\subsection{EKKS 的收敛条件}
EKKS 的收敛要求 \ref{eqs:ekks-omega} 的级数展开收敛,取矩阵的范数,有 
\begin{equation}
  \lVert\omega_n\rVert\le\sum_{k=0}^{\infty}\lVert QH_1P\rVert\cdot\left\lVert\frac{1}{E-QHQ}\right\rVert^{k+1}\cdot\lVert H_{n-1}^\text{eff}-E \rVert^k,
\end{equation}
为使 $\lVert\omega_n\rVert$ 有限,取 $n\to\infty$,需满足
\begin{equation}
  \left\lVert\frac{1}{E-QHQ}\right\rVert\cdot\lVert H_{n-1}^\text{eff}-E\rVert<1,
\end{equation}
应用一个数学定理,矩阵的范数大于等于它本征值绝对值中最大的,
\begin{equation}
  \lVert A\rVert\ge\max|\lambda_i|,
\end{equation}
因此得出 EKKS 的收敛条件 
\begin{equation}
  \left|\frac{E_p-E}{\varepsilon_q-E}\right|<1,\quad \text{for any } p=1,\ldots,d,\quad q=d+1,\ldots,D,\label{eqs:ekks-conv}
\end{equation}
其中 $E_p$ 为 $H^\text{eff}$ 的本征值,$\varepsilon_q$ 为 $QHQ$ 的本征值,但 $QHQ$ 的本征值不容易知道,因此这个判据在实际计算中无法使用。而且严格来说这个收敛条件既不充分也不必要 \cite{2011takayanagi2}。\cite{2011takayanagi2} 指出 $E$ 应该取在 $P$ 空间能谱的中间。
% 对于实际求解来说,$E$ 的参数有两个,$\hat{S}$-box 的 $E_1$ 可以根据对角的单粒子能确定,这就是有效哈密顿量的本征值;而 $\hat{Q}$-box 的 $E_2$ 应该用 KSHELL 计算两价核子体系的能谱确定。在壳模型计算中两价核子体系有 2p、2n、1p1n 三种,对每个原子核来说 KSHELL 中的矩阵应该只是 MBPT 得出的有效哈密顿量的块对角部分。最合理的方式应该是 2p、2n、1p1n 这三个原子核全算一下,所有算出来的能谱都是有效哈密顿量的本征值。

这个收敛条件看上去很像数学物理方法中关于泰勒展开收敛半径的结论:$f(z)$ 在 $a$ 处展开,$b$ 为距离 $a$ 最近的奇点,则 $f(z)$ 可以在圆 $|z-a|<|b-a|$ 内展开为泰勒级数,或者说在圆内泰勒级数收敛。正如之前的讨论,EKKS 中泰勒展开实际是对 $H^\text{BH}(E^{(n-1)}_i)$ 进行的,因此这里的 $z$ 就对应 $E^{(n-1)}_i$,也就是 $E_p$,$b$ 对应 $\varepsilon_q$,也就是 $\hat{Q}$ 可能的奇点,而 $a$ 就是展开点 $E$。

\section{$\hat{Q}$-box 的计算}
代码中以拉格朗日插值法求 $\hat{Q}$-box 的导数,相当于以插值法画出图像并求交点。作为示意,展示金师兄笔记中的示意图 \ref{fig:qbox}。实际算出来的 $\hat{Q}$-box 的极点为单粒子能的加减,而输入中给出的 shift1b 与 shift2b 两个量则是控制迭代的开始位置。如果初始位置取的离极点太近,就有可能出现发散的问题。为了增加收敛性,将 shift1b 与 shift2b 减小是可能的办法。
\begin{figure}[htbp]
  \centering
  \includegraphics[width=0.6\textwidth]{figure/qbox.png}
  \caption{$\hat{Q}(\epsilon)$ 找交点,因为实际方程为矩阵,交点对应的是要求的模型空间哈密顿量。}
  \label{fig:qbox}
\end{figure}

此外,真实的 $\hat{Q}$-box 在每个区间都是减函数,但是在有截断的计算中有可能算出来上升的部分,这时再插值也有可能插值出来发散的部分。为了解决这个问题,尝试用严格求导的方式来计算导数。

\subsection{精确求导}
二阶图有一个形如 $1/(x+a)$ 的能量分母,其 $n$ 阶导数的形式比较简单,是
\begin{equation}
  \left(\frac1{x+a}\right)^{(n)}=(-1)^n n!\frac1{(x+a)^{n+1}},
\end{equation}
而三阶图的能量分母为 $1/(x+a)(x+b)$,虽然可以拆成
\begin{equation}
  \frac{1}{(x+a)(x+b)}=\frac1{b-a}\left(\frac1{x+a}-\frac1{x+b}\right),
\end{equation}
但对于某些图如 diag-1bp-4,能量分母为 $(\epsilon+\epsilon_h-\epsilon_{p_1}-\epsilon_{p_2})(\epsilon+\epsilon_h-\epsilon_{p_3}-\epsilon_{p_4})$,在循环中 $p_3,p_4$ 如果与 $p_1,p_2$ 取成了相同的态,就会导致这一项求和的系数 $1/(b-a)$ 非常大,数值上出现问题。因此在代码中使用的导数公式为
\begin{align}
  \left[\frac{1}{(x+a)(x+b)}\right]^{(n)}&=(-1)^nn!\sum_{k=1}^{n+1}\frac1{(x+a)^k(x+b)^{n+2-k}}.
\end{align}
实际上也完全可以把 $n!$ 直接从二阶图与三阶图的导数公式中去掉,然后在之后的代码中也不再除掉 $n!$。

\subsection{Padé 近似}
\cite{2012coraggio} 中提到了用 Padé 近似估计级数收敛后的值,这里的级数指的是 $\hat{Q}$-box 展开到三阶图的这个 $1/(E-QHQ)$ 级数,并不是 EKK 的级数。也就是说可以用展开到三阶的级数去估计 $\hat{Q}$-box 的高阶项,公式为
\begin{equation}
  [2|1] = V_{Qbox}^0 + V_{Qbox}^1+ V_{Qbox}^2 \bigl( 1 - (V_{Qbox}^2)^{-1} \, V_{Qbox}^3 \bigr)^{-1}.
\end{equation} 
回顾一下 Padé 近似的目的,有一个级数
\begin{equation}
  F(x)=\sum_{n=0}^\infty F_nx^n,
\end{equation}
由于计算能力的限制,只能计算几个 $F(x)$ 的低阶系数,如果把泰勒级数截断到这么低阶的项,就可能与 $F(x)$ 差别很大。但又想把这个展开应用到更大 $x$ 的区域,此时 Padé 近似就有可能起到用处。将 $H$ 写为 $H=H_0+xH_1$,此时 
\begin{equation}
  \hat{Q}(\epsilon,x)=xPH_1P+x^2PH_1Q\frac1{\epsilon-QH_0Q-xQH_1Q}QH_1P,
\end{equation} 
将分式展开为
\begin{equation}
  \frac1{\epsilon-QH_0Q-xQH_1Q}=\sum_{n=0}^\infty\frac{(xQH_1Q)^n}{(\epsilon-QH_0Q)^{n+1}},
\end{equation}
因此可定义 
\begin{equation}
  F_2=PH_1Q\frac1{\epsilon-QH_0Q}QH_1P,\quad F_3=PH_1Q\frac1{\epsilon-QH_0Q}QH_1Q\frac1{\epsilon-QH_0Q}QH_1P,
\end{equation}
注意到 $F_2$ 是二阶图,$F_3$ 是三阶图,带回得到
\begin{equation}
  \hat{Q}(\epsilon,x)=F_1x+F_2x^2+F_3x^3,
\end{equation}
其中 $F_1\equiv PH_1P$,也就是一阶图。
对这个展开式应用 Padé 近似,并取 $x=1$ 即可。

但这样还不够,不管是 EKK 还是 ELS 都需要 $\hat{Q}$-box 的高阶导数,在原代码中通过数值差分求导数,自然可以得到这个 Padé 近似之后的 $\hat{Q}$-box 的高阶导数。但是改为严格求导之后就不能这么做了,只能对 Padé 近似式求导。虽然从数学上直接对 Padé 近似式求导不知道有什么意义,但至少在数值上与直接差分肯定是一样的。因此还是需要推导这个 Padé 近似式的导数。

为了方便标记,分别用 $q_1,q_2,q_3$ 表示 $\hat{Q}$-box 的一二三阶图,需要求导的项就是
\begin{equation}
  f=q_2/(1-q_3/q_2)=q_2^2/(q_2-q_3),
\end{equation}
为了求 $f$ 的高阶导数,改写为
\begin{equation}
  (q_2-q_3)f=q_2^2,
\end{equation}
用莱布尼茨公式对两侧求 $n$ 阶导数,首先看右侧,
\begin{equation}
  (q_2^2)^{(1)}=2q_2q_2^{(1)},
\end{equation}
因此 
\begin{equation}
  (q_2^2)^{(n)}=2(q_2q_2^{(1)})^{(n-1)}=2\sum_{k=0}^{n-1}C_{n-1}^kq_2^{(n-k-1)}q_2^{(k+1)},\quad n\ge1,
\end{equation}
而左侧为
\begin{equation}
  \sum_{k=0}^{n-1}C_n^k[q_2^{(n-k)}-q_3^{(n-k)}]f^{(k)}+(q_2-q_3)f^{(n)},
\end{equation}
因此
\begin{equation}
  f^{(n)}=\frac1{q_2-q_3}\sum_{k=0}^{n-1}\left\{2C_{n-1}^kq_2^{(n-k-1)}q_2^{(k+1)}-C_n^k[q_2^{(n-k)}-q_3^{(n-k)}]f^{(k)}\right\},
\end{equation}
再除以 $n!$ 换为 $f_n$,
\begin{equation}
  f_n=\frac1{q_2-q_3}\sum_{k=0}^{n-1}\left\{2\frac{k+1}{n}q_{2,n-k-1}q_{2,k+1}-[q_{2,n-k}-q_{3,n-k}]f_k\right\}.
\end{equation}

需要澄清的是,在 MBPT 中除了 $\hat{Q}$-box 展开的级数,使用 EKK 方法时还有一个泰勒级数,同样将这个级数截断到了有限阶。理论上说这也可以用 Padé 近似去更好地处理,但是对 \ce{^38K} 用 EKK 与 ELS 的实际计算发现,在 $E$ 选取相同时结果也几乎相同,而 ELS 并不涉及级数的问题,是个严格的迭代过程,看上去 EKK 的级数展开是收敛的,应该不存在截断到有限项与原函数差别较大的问题。结果与实验值差别较大应该有其他原因。

\subsection{$\hat{Z}$-vertex}
为了解决 $\hat{Q}$-box 的极点发散问题,引入了 $\hat{Z}$-vertex,
\begin{equation}
  \hat{Z}(\epsilon)=\frac{1}{1-\hat{Q}_1(\epsilon)}[\hat{Q}(\epsilon)-\hat{Q}_1(\epsilon)P(\epsilon-H_0)P].\label{eqs:z-vertex}
\end{equation}
将 \ref{eqs:n-step-heff} 在 $P$ 空间对角化,对应的 \ref{eqs:n-1 diagonalization} 改写为
\begin{equation}
  [PH_0P+R^{(n-1)}]|\phi_i^{(n)}\rangle=E_i^{(n)}|\phi_i^{(n)}\rangle,\quad i=1,2,\ldots,d,
\end{equation}
其中
\begin{equation}
  R^{(n-1)}=\sum_{j=1}^d\hat{Q}(E_j^{(n-1)})|\phi_j^{(n-1)}\rangle\langle\widetilde{\phi}_j^{(n-1)}|,
\end{equation}
$PH_0P$ 这一项与被求和的 $j$ 无关,且本身就在 $P$ 空间,根据 \ref{eqs:ekkd-norm},$P$ 空间的正交基不是归一化到单位算符,而是投影算符,考虑到 $P^2=P$,因此 $PH_0P$ 这一项会单列出来。

当迭代收敛时,也就是 $n-1$ 上标与 $n$ 上标相等,得到自洽条件为
\begin{equation}
  (E_i-H_0)|\phi_i\rangle=\hat{Q}(E_i)|\phi_i\rangle,
\end{equation}
也可以写为
\begin{equation}
  (E_i(\epsilon)-H_0)|\phi_i\rangle=\hat{Q}(\epsilon)|\phi_i\rangle,\quad \epsilon=E_i(\epsilon).\label{eqs:z-vertex-con1}
\end{equation}

定义了 $\hat{Z}$-vertex 之后,在 $\epsilon=E_i(\epsilon)$ 时,有
\begin{equation}
  \hat{Z}(\epsilon)|\phi_i\rangle=\hat{Q}(\epsilon)|\phi_i\rangle,\label{eqs:z-vertex-con2}
\end{equation}
也就是说迭代收敛之后,$\hat{Q}$-box 与 $\hat{Z}$-vertex 作用于收敛的本征态上的结果是一样的。因此 \ref{eqs:n-step-heff} 可用 $\hat{Z}$-vertex 定义为
\begin{equation}
  H_{n}^{\mathrm{eff}}=\sum_{i=1}^d [PH_0P+\hat{Z}(E_i^{(n-1)})]|\phi_i^{(n-1)}\rangle\langle\widetilde{\phi}_i^{(n-1)}|,
\end{equation}
虽然每一步的迭代中用 $\hat{Q}$-box 与 $\hat{Z}$-vertex 构建的 $H_{n}^{\mathrm{eff}}$ 是不一样的,但收敛之后是一样的。这样就在保证结果相同的前提下避免了 $\hat{Q}$-box 的极点问题。

为了加快计算,同样可用泰勒展开的方式计算迭代每一步的 $\hat{Z}$-vertex。这就需要计算 $\hat{Z}$-vertex 的各阶导数,用莱布尼茨公式,
\begin{equation}
  (u\cdot v)^{(n)}=\sum_{k=0}^nC_n^ku^{(n-k)}v^{(k)},
\end{equation}
将 $\hat{Z}$-vertex 拆分为两个函数的乘积,
\begin{equation}
  f(\epsilon)=\frac{1}{1-\hat{Q}_1(\epsilon)},\quad g(\epsilon)=\hat{Q}(\epsilon)-\hat{Q}_1(\epsilon)P(\epsilon-H_0)P,
\end{equation}
$f(\epsilon)$ 的高阶导数用 Mathematica 算了一下也没看出规律来,所以我们这样计算:对
\begin{equation}
  f(\epsilon)=1+f(\epsilon)\hat{Q}_1(\epsilon),
\end{equation}
两侧求 $k$ 阶导数,有
\begin{equation}
  f^{(k)}(\epsilon)=\frac{1}{1-\hat{Q}^{(1)}(\epsilon)}\sum_{i=1}^k C_k^if^{(k-i)}(\epsilon)\hat{Q}^{(i+1)}(\epsilon),
\end{equation}
注意 $\hat{Q}^{(k)}(\epsilon)=k!\hat{Q}_k(\epsilon)$ 表示导数本身,因为代码中计算的是 $\hat{Q}$-box 的各阶导数,而没有把 $k!$ 除掉。这样就可以从最低阶开始逐阶递推得到高阶导数。代入 $k=1,2$ 验证,有 
\begin{equation}
  f^{(1)}(\epsilon)=\frac{\hat{Q}^{(2)}(\epsilon)}{[1-\hat{Q}^{(1)}(\epsilon)]^2},\quad f^{(2)}(\epsilon)=\frac{2[\hat{Q}^{(2)}(\epsilon)]^2}{[1-\hat{Q}^{(1)}(\epsilon)]^3}+\frac{\hat{Q}^{(3)}(\epsilon)}{[1-\hat{Q}^{(1)}(\epsilon)]^2},
\end{equation}
而 $g(\epsilon)$ 的 $k$ 阶导数为
\begin{equation}
  g^{(k)}(\epsilon)=-\hat{Q}^{(k+1)}(\epsilon)P(\epsilon-H_0)P-(k-1)\hat{Q}^{(k)}(\epsilon),
\end{equation}
代入 $k=0,1$ 也是完全成立的。因此 $\hat{Z}$-vertex 的 $n$ 阶导数为 
\begin{equation}
  \hat{Z}^{(n)}(\epsilon)=\sum_{k=0}^nC_n^kf^{(n-k)}(\epsilon)g^{(k)}(\epsilon),
\end{equation}
代入 $k=1$ 验证,
\begin{equation}
  \hat{Z}^{(1)}(\epsilon)=\frac{2\hat{Q}_2(\epsilon)}{[1-\hat{Q}_1(\epsilon)]^2}[\hat{Q}(\epsilon)-\hat{Q}_1(\epsilon)P(\epsilon-H_0)P]-\frac{2\hat{Q}_2(\epsilon)}{1-\hat{Q}_1(\epsilon)}P(\epsilon-H_0)P,
\end{equation}
与 \cite{2014dong} 给出的公式一致。也可以写为
\begin{equation}
  \hat{Z}^{(1)}(\epsilon)=\frac{2\hat{Q}_2(\epsilon)}{1-\hat{Q}_1(\epsilon)}[\hat{Z}(\epsilon)-P(\epsilon-H_0)P],
\end{equation}
由 \ref{eqs:z-vertex-con1} 与 \ref{eqs:z-vertex-con2},可知收敛时
\begin{equation}
  \hat{Z}^{(1)}(\epsilon)|\phi_i\rangle=0,\quad \epsilon=E_i(\epsilon).\label{eqs:z-vertex-1deriv}
\end{equation}
对于 $\hat{S}$-box,是对每个对角元进行迭代,因此本征波函数就是一个数,也就是 $\hat{Z}^{(1)}(\epsilon)=0$,由 \ref{eqs:z-vertex} 与 \ref{eqs:z-vertex-1deriv},得到 $\hat{Z}(\epsilon)=\hat{Q}(\epsilon)$,从计算结果中也可以看出,在迭代的最后一步,这个关系确实成立,但没收敛时两者有区别。
