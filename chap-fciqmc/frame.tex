\section{理论推导}
量子蒙特卡洛大致可以分为两类,一类是变分蒙特卡洛(VMC),其基础是变分法,从试探波函数出发,通过对波函数的变分优化寻找体系的基态,计算能量期望值的过程需要计算高维积分,VMC 用蒙特卡洛模拟的方式计算高维积分。另一类是投影量子蒙特卡洛,包括扩散蒙特卡洛(DMC),格林函数蒙特卡洛(GFMC),辅助场蒙特卡洛等,投影的本质就是解薛定谔方程。

全组态相互作用量子蒙特卡洛(FCIQMC)也是投影量子蒙特卡洛方法,出发点是虚时薛定谔方程,虽然将虚时间 $\tau$ 定义为 $it$,然而虚时薛定谔方程的本质是原本薛定谔方程的 Wick 转动,在转动之后虚时间也是实数。虚时薛定谔方程为
\begin{equation}
    \frac{\partial}{\partial\tau}\psi(\tau) = -H\psi(\tau),
\end{equation}
通过 Wick 转动变为虚时演化的目的在于,将 $e^{-iHt}$ 因子导致的振荡变为形如 $e^{-\tau H}$ 的指数衰减。这样就能够通过指数衰减将演化初始波函数中的非基态成分去掉,只留下基态。虚时薛定谔方程的解是
\begin{equation}
    \psi(\tau)=e^{-\tau H}\psi(\tau=0),
\end{equation}
虽然这里写的是等于,但因为波函数总是可以添加相因子的,因此其实可以写为
\begin{equation}
    \psi(\tau)\propto e^{-\tau H}\psi(\tau=0),
\end{equation}
在后面也会额外添加相因子,不过目前还是先从等于的这个式子出发,这样才能更好地理解后面添加相因子的意义。考虑 $H$ 的本征方程
\begin{equation}
    H\Psi_n=E_n\Psi_n,
\end{equation}
将虚时为 0 时的任意初始波函数用这组基矢展开,
\begin{equation}
    \psi(\tau=0)=\sum_nc_n\Psi_n,
\end{equation}
因此
\begin{equation}
    \psi(\tau)=e^{-\tau H}\sum_nc_n\Psi_n=\sum_nc_ne^{-\tau E_n}\Psi_n,
\end{equation}
虚时演化到 $\tau\to\infty$,因为 $E_0<E_1<E_2<\ldots$,
\begin{equation}
    e^{-\tau E_n}/e^{-\tau E_0}=e^{-\tau(E_n-E_0)}\to0,\quad n>0,
\end{equation}
此时主导的一定是基态波函数的系数,也即
\begin{equation}
    \lim_{\tau\to\infty}\psi(\tau)=c_0e^{-\tau E_0}\Psi_0.
\end{equation}
这还不完全是想要的基态波函数 $\Psi_0$,$c_0$ 可以直接归一化,但需要注意 $c_0\ne0$,也就是选取的演化初始波函数必须包含基态的成分,与基态完全正交显然是演化不出基态的。另外多出的 $e^{-\tau E_0}$ 因子也可以在虚时演化算符的定义中将其去掉,也就是前面提到的,可以把虚时薛定谔方程的解添加相因子,写为
\begin{equation}
    \psi(\tau)=e^{-\tau(H-E_0)}\psi(\tau=0),
\end{equation}
这种写法等价于如下虚时薛定谔方程
\begin{equation}
    -\odv{}{\tau}\psi(\tau)=(H-E_0)\psi(\tau).
\end{equation}

接下来在组态空间中进行虚时演化。组态空间的基矢是 Slater 行列式 \(\ket{D_i}\),将 $\psi(\tau)$ 在组态空间展开为
\begin{equation}
    \psi(\tau)=\sum_iC_i(\tau)\ket{D_i},
\end{equation}
代入虚时薛定谔方程得到
\begin{equation}
    -\odv{}{\tau}C_i(\tau)=\sum_j(H_{ij}-S\delta_{ij})C_j(\tau), 
\end{equation}
由于并不能在计算之前就知道 \(E_0\) 是多少,因此在实际计算时需要把 \(E_0\) 换为一个自适应调节的位移参数 \(S\)。

由于展开系数 \(C_i\) 有正也有负,无法将其分布情况转换为概率分布,因此 FCIQMC 将 \(N_i\) 个 walker 放在组态 \(\ket{D_i}\) 上,这些 walker 是带有符号的,因此可以表示负的展开系数。要求 walker 数目 \(N_i\) 与 \(C_i\) 成正比,因此 \(N_i\) 满足的演化方程离散化为
\begin{equation}
    -\adv{N_i}{\tau}=\sum_j(H_{ij}-S\delta_{ij})N_j,\label{eqs:fciqmc-master1}
\end{equation}
问题的关键就变成了设计一套算法,使 \(N_i\) 按照 \ref{eqs:fciqmc-master1} 演化。可以将 \ref{eqs:fciqmc-master1} 拆分为非对角项与对角项两部分,即
\begin{equation}
    -\adv{N_i}{\tau}=(H_{ii}-S)N_i+\sum_{j\ne i}H_{ij}N_j.\label{eqs:fciqmc-master2}
\end{equation}

先看非对角部分。对于每个 \(\ket{D_i}\),都有一系列 \(\ket{D_j}\) 与其相连,在 FCI 计算中这样的行列式非常多,不可能全纳入计算。这里就是 FCIQMC 的随机性的体现:通过蒙特卡洛方法采样波函数,保留重要的组态,以大幅度降低波函数的维数。简单来说,对于组态 \(\ket{D_i}\) 上的每个 walker,将其以一定的概率激发到某个与之相连的组态 \(\ket{D_j}\) 上,即在 \(\ket{D_j}\) 上以一定的概率生成 walker。这种表述与 \ref{eqs:fciqmc-master2} 的 \(i,j\) 记号其实是反着的,但 \cite{2009booth} 等文献都是把公式写为 \ref{eqs:fciqmc-master2},在语言表述中也是反过来。FCIQMC walkers 随机游走的详细过程是,目前演化到虚时间 \(\tau\),全空间的 walkers 分布为 \(\bm{N}(\tau)\),对每个有占据的行列式 \(\ket{D_i}\) 上的 \(N_i\) 个 walkers 中的每一个,将其以一定的概率激发到与 \(\ket{D_i}\) 相连的 \(\ket{D_j}\) 上,这个过程写为
\begin{equation}
    \Delta N_{j\gets i}=p_\text{spawn}(j|i)=\frac{\Delta
    \tau|H_{ji}|}{p_\text{gen}(j|i)},
\end{equation}
在确定性的对角化方法中,\(\ket{D_i}\) 的波函数系数 \(C_i\) 对所有其他的 \(C_j\) 都有贡献,贡献的强度由 \(H_{ij}\) 衡量。而在随机算法中,用 walker 数目 \(N_i\) 随机采样波函数的系数,目标是在每次演化中投影出重要的组态,以控制内存。因此不能在一步演化中完整考虑与 \(\ket{D_i}\) 相连的每一个 \(\ket{D_j}\),而是对 \(N_i\) 个 walkers 中的每个进行随机激发,就能在多步演化之后,从统计意义上达到 FCI 的效果。从 \(\ket{D_i}\) 激发到 \(\ket{D_j}\) 的概率为 \(p_\text{gen}(j|i)\),一种最简单的计算方法就是按照轨道占据等概率激发,当然还有更高效的激发算法,这会在后续讨论。激发概率出现在分母上相当于条件概率,walker 先选择激发到哪个 \(\ket{D_j}\),再按照 \ref{eqs:fciqmc-master2} 的要求于 \(\ket{D_j}\) 上生成对应数量的 walker。以上过程称为 spawning,实际计算中需要对 \(\bm{N}(\tau)\) 的每个行列式上的每个 walker 都进行以上的 spawning。

再看对角部分。这意味着在 \(\tau\to\tau+\Delta\tau\) 的演化中,\(\ket{D_i}\) 上的 \(N_i\) 个 walker 以
\begin{equation}
    p_\text{death}(i)=\Delta\tau(H_{ii}-S)
\end{equation}
的概率死亡。如果 \(p_\text{death}(i)<0\),则是以 \(-p_\text{death}(i)\) 的概率复制。由于以上演化全是在 \(\ket{D_i}\) 上进行的,只涉及 \(\ket{D_i}\) 上的 walkers,因此 \(N_i\) 死亡或克隆的数量就是
\begin{equation}
    \Delta N_i=-(H_{ii}-S)N_i\Delta\tau.
\end{equation}
这个过程称为 diagonal death/cloning。

这两个过程完成之后,得到了 \(\tau+\Delta\tau\) 时刻的一组新 walkers 分布。由于 walkers 都是自带符号的,接下来进行 annihilation 过程把符号相反的 walker 湮灭掉。该步骤是为了控制 walkers 的总数,也是 FCIQMC 方法处理费米子符号问题的关键步骤。

\section{计算步骤}
\subsection{预热}
FCIQMC 计算本质是对某个波函数虚时投影,提取出基态成分。也就是说需要将一定数量的 walkers 放在 \(\psi(\tau=0)\) 上作为计算的起点。


\section{费米子符号问题}
QMC 的本质是用概率分布模拟波函数。但量子力学玻恩的概率波诠释是波函数振幅(概率幅)的模方表示概率。然而 QMC 直接处理的是波函数的概率幅本身,对于费米子,由于需满足交换反对称性,波函数的概率幅一定是有正有负的(甚至还可能是复数),这就会导致 QMC 计算可观测量的平均值时,
\begin{equation}
    \expval{O}=\frac{\sum_i w_iO_i}{\sum_i w_i},
\end{equation}
权重 \(w_i\) 只采样了一部分概率幅,所以可能使分母的正负相抵,统计时平均值出现剧烈振荡,或者说统计噪声。另外,费米子符号问题还有另一种表现形式,投影量子蒙特卡洛方法求解的是薛定谔方程,但如果像 DMC 那样不在反对称化的空间中进行求解,可能无法保证费米子波函数的符号结构,导致投影出哈密顿量的无节点的最低能量解,该最低能量波函数不满足费米子的反对称性,这称为“Boson catastrophe”,也是费米子符号问题的一种体现。

通常解决费米子符号问题的方式是固定节点近似或约束路径方法,这些方法都需要比较好的试探波函数(或称为指导波函数)对波函数的节点结构进行限制,也是系统误差的主要来源。

而 FCIQMC 在反对称化的子空间,也就是组态空间进行,能够避免计算出玻色子基态。但仅仅如此并不能完全避免符号问题,因为即使在组态空间计算也有负的概率幅。在此基础上,符号相反的 walker 湮灭有助于提高体系的信噪比 \cite{2009booth},进一步压制符号问题;湮灭操作并非 FCIQMC 的首创,但相比在连续空间的 GFMC 与 DMC walker 的湮灭,离散的组态进行湮灭非常简单与自然。

然而费米子符号问题也没有被 FCIQMC 完全解决。最原始的 FCIQMC 算法存在的问题是,需要 walker 数目达到一定值才能稳定地收敛到基态。虚时演化收敛需要的最小 walker 数目与 Hilbert 空间的维数成比例,这会导致 FCIQMC 需要的计算量并不比 FCI 小多少。FCIQMC 虚时演化收敛需要的 walker 数目很大,这是费米子符号问题在 FCIQMC 这个方法中的表现形式。FCIQMC 演化收敛需要相当数量的 walkers,这是因为 \(\ket{\Psi_0}\) 与 \(-\ket{\Psi_0}\) 都是体系允许的基态波函数,然而在随机游走的过程中,可能会激发出属于 \(-\ket{\Psi_0}\) 的 walkers,导致大部分 walkers 是 \(\ket{\Psi_0}\) 的,而小部分 walkers 是 \(-\ket{\Psi_0}\) 的。这两部分 walkers 对基态波函数贡献的符号相反,且这种差别会通过指数的投影算符继续向后传播,导致随机产生的少数属于 \(-\ket{\Psi_0}\) 的 walkers 的效应被迅速放大,演化难以收敛。

\section{起始子近似}
体系波函数只有小部分的主导组态的概率幅绝对值比较大,绝大部分组态都是少占据或不占据的。在随机游走中,如果某个 walker spawn 到不被占据的行列式上,这个新 walker 的符号就完全由原来的 walker 决定,这就会出现符号波动,且会被指数投影迅速向之后的演化传播。也就是说,为了压制符号问题,限制可能符号不对的行列式上的 walkers 向未占据的行列式上 spawning 是一个直接的思路,这就是 initiator-FCIQMC(\textit{i}-FCIQMC)。

将 walker 数目大于给定阈值 \(n_a\) 的行列式称为 initiator 行列式,这些行列式的 walker 数目多,因此可认为符号是正确的,这些是 initiator 的 walkers 可以向未占据的行列式激发。不是 initiator 的 walkers 则不能向未占据的行列式激发,这能防止某个符号错误的 walker 主导与其相连的行列式的符号,并将错误的符号继续传播下去;非 initiator 的 walker 可以激发到已经有占据的行列式上,这些行列式上已经存在的 walkers 能够把这个符号错误的 walker 湮灭掉,阻止统计噪声的出现。

公式

已经证明 \textit{i}-FCIQMC 在任意的 walker 数目下都能实现稳定的演化 \cite{2019ghanem-as},但这是以引入系统误差为代价的。显然 initiator 近似相当于对真实的哈密顿量施加了截断

\subsection{Adaptive shift FCIQMC}
需要注意的是,adaptive shift 方法并不是必须要进行的,如果计算能力足够将所研究的体系用 \textit{i}-FCIQMC 算到对 walker 数收敛,系统误差就会被消除。也就是说 initiator 近似引入的系统误差并不是方法本身的缺陷。Adaptive shift 方法是计算能力不足以算收敛时引入的一种改进方法,其本质是用微扰将丢掉的部分补回来。