作为一种新引入核物理的方法,FCIQMC 有大量的计算细节还不明确怎么处理是最好的,需要在测试中探索最合适的计算范式。本章结合计算结果,讨论的问题包括:对简单模型的基准测试,计算的参数如何设置,计算的收敛性如何,结果应该如何统计分析等。

\section{基准测试}
FCI 计算能精确对角化的小空间体系自然可以作为 FCIQMC 的基准测试。除此之外,量子化学对 FCIQMC 的测试,研究 FCIQMC 的行为与特征很多不是对真实的电子体系进行计算,而是使用 Hubbard 模型,Heisenberg 模型等。这些模型虽然不是都有严格解,但其物理图像简单,有物理意义明确的可调参数,并且用其他高精度方法计算也很简单。接下来对这些模型的哈密顿量进行简要介绍。
\subsection{Richardson pairing 模型}


\subsection{Hubbard 模型}

\section{计算设置}
原则上第一性原理计算是不需要引入额外参数的,但实际计算中一些需要手动调整的量或设置也是必不可少的,比如核力的 \(\hbar\omega\) 与 \(e_\text{max}\),模型空间,以及 FCIQMC 计算中会设置的虚时间步长 \(\dd\tau\),演化步数,walker 数目 \(N_w\),以及单激发、双激发概率,非整数 walker 数目截断阈值等很多细节的设置。虽然 \textit{ab initio} 计算不愿意称之为参数,但貌似也没有更合适的代替名词。不过这里还是将其广泛地称为“计算设置”,因为至少选模型空间轨道的这种设置不太能用某个“数”概括。
\subsection{虚时演化步长 \(\dd\tau\)}

\subsection{虚时演化步数}
投影量子蒙特卡洛方法对体系进行随机的虚时演化,并不能得到能量或波函数的确定值,只能在不同的虚时间下进行采样得到一批数据,需要对这些数据进行统计处理。以能量的计算为例,任何演化都不会在一开始就达到平衡或稳定,在稳定演化之前的波函数还混杂了大量非基态的成分。进行一定时间的虚时演化之后,波函数会调整为正确的基态波函数,能量也会在真实的基态能量附近上下波动,对这些稳定演化的数据进行采样与统计平均,才是正确的。

在 FCIQMC 中,演化步数需要在计算前进行设置。步数的设置既需要保证演化能够稳定,也需要在稳定后保证有足够的数据进行统计分析,同时为了节约计算时间,也不宜过多。图 \ref{fig:o16-evol-140k} 演化了 14 万步发现,增大演化步数并不会显著改善结果的统计误差,只要演化稳定了,能量的波动就在一个确定的范围内,并没有出现演化步数越多标准差越小的现象,要想提高统计精度,更应该增大 walker 数目 \(N_w\)。

\begin{figure}[!htbp]
  \centering
  \includegraphics[width=0.5\textwidth]{figure/fciqmc/O16_evol_140k.png}
  \caption{\ce{^16O} 虚时演化了 14 万步的过程。}
  \label{fig:o16-evol-140k}
\end{figure}

那演化步数应该取多少呢?在能接受的计算时间内自然是步数越多越安全,不过这也不是个过于重要的问题,因为 \(E(\tau)\) 的波动本身就不大,用几千个数据点统计分析已经足够了。这段讨论只是为了说明,更多的演化步数并不会改善结果的统计误差,只要不会少到影响统计分析即可。演化步数对统计分析的影响将会在后文进一步讨论。


\section{收敛性}
\subsection{虚时演化的稳定性}
如前所述,虚时演化的步数设置必须保证演化稳定。因此如何判断什么时刻进入稳定演化,以及哪些数据是可用的就成为了重要问题。最简单的做法就是画出 \(E\) 随着 \(\tau\) 的演化图,观察进入稳定演化的位置。但在 FCIQMC 中,使用热浴采样(heat-bath sampling)后能量会在演化的一开始就很稳定,很难看出在什么时刻演化变得稳定。不过在 FCIQMC 中还存在一个自适应调整的位移 \(S\) 用于控制 walker 数目 \(N_w\),在演化稳定之后 \(S\) 也会调整到基态能量附近,这样才能使 \(N_w\) 与 \(E\) 均稳定。由于 \(S(\tau)\) 的波动是比 \(E(\tau)\) 大的,因此 \(S(\tau)\) 也许是一个更好的判断演化何时稳定的标志。

用 N\(^2\)LO\(_{\text{opt}}\) 核力,\(e_\text{max}=8\),\(\hbar\omega=16\) MeV,以 adaptive-shift 方法,取 \(n_a=3,\Delta=0.5E_0\) 计算了 \ce{^16O},改变 \(N_w\) 数目,虚时演化如图 \ref{fig:o16-evol-nw} 所示。可以看出,\(S(\tau)\) 的波动确实比 \(E(\tau)\) 大,而且随着 \(N_w\) 的增加,两个量的波动都变小。这也是很好解释的,walker 数目越多,对基态能量有贡献的重要组态就能被更充分地包含进基态波函数中,\(S\) 与 \(E\) 自然会更加稳定。图 \ref{fig:o16-evol-nw} 未画 \(N_w<10^6\) 的结果,因为 walker 数目过少时 \(S(\tau)\) 的波动非常大。 
\begin{figure}[!htbp]
  \centering
  \includegraphics[width=1.0\textwidth]{figure/fciqmc/O16_evol_Nw.png}
  \caption{不同 \(N_w\) 下 \ce{^16O} 的虚时演化 20000 步的过程。}
  \label{fig:o16-evol-nw}
\end{figure}

\(E(\tau)\) 的演化比较稳定,没有很明显的波动,只是在演化初期有明显地下降,下降到一平台后就会围绕该平台稳定地波动。而 \(S(\tau)\) 则会先下降到低于基态能量,再调整到基态能量附近,围绕基态能量波动。相比 \(E(\tau)\),\(S(\tau)\) 稳定的位置更容易看出来,可以作为判断虚时演化稳定的指标。

从以上的计算中还可以看出 \(N_w\) 越大,\(S(\tau)\) 稳定所需要的步数越多。图 \ref{fig:o16-evol-nw} 中 \(N_w=10^6\) 经过 1500 步演化 \(S(\tau)\) 即稳定了,而 \(N_w=5\times10^6\) 则需要大约 3000 步才能稳定,增大到 \(N_w=10^7\) 则需要大约 5000 步才能稳定。

不过在实际计算中,使用更大的 \(N_w\) 得到的结果才能更好地降低系统误差,此时要想得到稳定的演化就需要更多的步数,比如 \(N_w=10^8\) 时演化 4000 步,\(S\) 还是显著低于 \(E\) 的,并未达到平台。但 \(N_w\) 越大,每一步的计算时间也越长。对此提出两种可能的解决方案。一种比较简单的做法是考虑到 \(N_w\) 较大时 \(E(\tau)\) 的波动本身就比较小,比如图 \ref{fig:o16-evol-nw} 中 \(N_w=10^7\) 时演化到 1500 步时就看不到 \(E(\tau)\) 在演化初期的下降了。这种做法比较简单,但让人担忧的是还是确定不了 \(E(\tau)\) 到底有没有真的稳定。因此另一种方案的设想是先用较小的 \(N_w\) 进行预演化,待 \(S(\tau)\) 稳定后以该波函数为初始波函数,用较大的 \(N_w\) 再次演化,这么做有可能用更少的步数就使得大 \(N_w\) 下的 \(S(\tau)\) 进入稳定阶段。

\section{块分析}