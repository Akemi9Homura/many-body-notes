壳模型非常基础,但仍有一些概念需要澄清,不要混淆。最早的壳模型本质上就是平均场与独立粒子近似,这个模型的图景是有一组单粒子轨道,将核子往这些单粒子轨道中填充。这组单粒子能级自然是求解薛定谔方程
\begin{equation}
  H\psi=\left[-\frac{\hbar^2}{2M}\nabla^2+V(\bm{r})\right]\psi=E\psi
\end{equation}
得到的,其中 $V(\bm{r})$ 就是一个核子感受到的其他核子的平均场。这个平均场的选取方式有两种,一是手动选取某种唯象的单粒子势,比如谐振子势或 Woods-Saxon 势,尤其是 Woods-Saxon 势,势的形状与原子核的密度分布类似,因此效果也比较好。手动选取的势的参数是通过实验给出的单粒子能级确定的。另一种方式就是后续会讨论的 Hartree-Fock 方法,从核子之间的两体相互作用出发,用自洽迭代的方式求出平均场,这样得到的就是 HF 单粒子基。

最早的壳模型只有单粒子能级,但如果停留在独立粒子近似上,壳模型只能解释幻数、双闭壳核基态以及外加一个粒子或一个空穴核的一部分单粒子能级,甚至连偶偶核基态为 $0^+$ 都无法解释。这是因为平均场无论选取得如何恰当,也不能完全抵消核子之间的相互作用,这部分没有被包含进平均场的相互作用称为剩余相互作用。有了剩余相互作用之后就会出现组态混合,此时称为“组态相互作用壳模型”,现在提到的壳模型一般都是组态相互作用壳模型,但要分清楚最早发展的壳模型并没有剩余相互作用。

剩余相互作用既可以拟合,也可以由核力推导出来。后者可以称为 \textit{ab initio} 壳模型 \cite{2019stroberg}(虽然我在裴老师的课上听到这个术语感觉是编的,但事实证明是我浅薄了,这个术语没有问题且很准确,就是指以 \textit{ab initio} 方法从现实核力获得壳模型相互作用的组态相互作用壳模型),比如 VS-IMSRG,NCSM,$\hat{Q}$-box 等目前常用的方法。

组态相互作用壳模型隐含的思想是,通过组态混合考虑的超出平均场的关联。还有另一种考虑超出平均场关联的思想是打破体系的对称性,比如形变壳模型,HFB,Nilsson 哈密顿量等。