Hartree-Fock 方法是一种基础的平均场方法,其基本思想是每个粒子在其他粒子产生的平均场中进行独立运动。HF 方法的前身是 Hartree 方法,没有考虑多粒子波函数的交换对称性,而 Hartree-Fock 方法则是认为 $N$ 粒子体系的波函数是每个粒子的单粒子波函数构成的 Slater 行列式,通过变分法得到 HF 自洽场方程,并进一步得到每一个单粒子波函数。再次强调,HF 方法只能得到单粒子波函数,得不到除了指定的 Slater 行列式之外的其他任何多体态。用一个 Slater 行列式给出的结果是很不对的,考虑的关联也只限于平均场层次,因此现在 HF 已经不是一种值得拿出来讨论的多体方法了。在 HF 的基础上考虑关联的方法称为后 HF(post-HF)方法。后 HF 的意思是这些方法考虑了超出 HF 的关联,而不是说必须在 HF 基下进行。

HF 方法目前常用于形成一组完备的单粒子基——HF 基。虽然在计算中取定的 Slater 行列式只占据了一部分,但没被占据的单粒子轨道也是可以通过 HF 方法得出的,因此单粒子基是完备的。在具体的计算中,由于核力是写在谐振子(HO)基下的,所以 HF 方程是在 j-scheme 的 HO 基下进行求解,求解出的 HF 基是 HO 基的叠加,然后再将 HO 基的哈密顿量转换到 HF 基,进行后续的多体计算。正如前面提到的,哈密顿量是分块对角的,每个对角块是一个 channel,因此 HF 方程的求解也是分 channel 进行的,每个 HF 基仅仅由同一 channel 内的各 HO 基叠加得到。注意 HO 基考虑了自旋-轨道耦合,因此也有 $j$ 量子数,比如 HF 基的 $0s_{1/2}$,是由 HO 基的 $0s_{1/2},1s_{1/2},2s_{1/s},\ldots$ 等轨道叠加得到的。