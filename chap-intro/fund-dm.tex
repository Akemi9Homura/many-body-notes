给定某个算符 \(\hat{O}^\lambda_\mu\) 以及波函数 \(|\Psi_i\rangle\) 与 \(|\Psi_f\rangle\),两个波函数都是角动量的本征态,计算可观测量可以直接计算 \(\langle \Psi_f|\hat{O}^\lambda_\mu|\Psi_i\rangle\)。波函数绝大多数都是组态空间的,是一系列行列式的叠加,为了计算出非 0 的可观测量,必须保证 \(\mu=M_f-M_i\),由于在壳模型计算中通常会限制偶 \(A\) 核的 \(M=0\),奇 \(A\) 核的 \(M=1/2\),因此需要 \(\mu=0\) 的球张量算符的分量。这可以通过 \ref{eqs:wigner-eckart} 实现。因为在 \cite{2007suhonen} 等教材中,给出的电磁跃迁等算符都是约化矩阵元,要想得到实际被夹在波函数之间的算符,就需要 Wigner-Eckart 定理。

但是直接用 \(m\)-scheme 的波函数夹算符计算可观测量会很慢,计算的量级是整个组态空间的维数。如果更改算符,就需要用波函数重新再计算一次。另外波函数的存储也很占用内存。因此,很多壳模型程序如 KSHELL,BIGSTICK 等选择先用波函数计算密度矩阵并输出,不是直接在程序内部处理算符信息,而是将波函数的信息与算符的信息分离开。密度矩阵的规模是由单粒子基决定的,因此不管是计算还是储存都要更方便。

\subsection{密度算符的基本理论}
态 \(\ket{\Psi}\) 的密度算符定义为
\begin{equation}
  \hat{\rho}=\ket{\Psi}\bra{\Psi},\label{eqs:density-matrix-pure}
\end{equation}
这里的 \(\ket{\Psi}\) 是纯态。纯态指的是能用确定的态矢量描述的态,可以不是某个力学量完全集的共同本征态。比如组态空间的某一个组态对应的态 \(\ket{D}\) 不是 \(H\) 的本征态,也就不是定态,如果制备出体系初态位于 \(\ket{D}\) 上,它会进行时间演化。

而密度算符更重要的意义是为了处理混态。混态无法用单一的态矢量描述,只能用密度算符描述混态,
\begin{equation}
  \hat{\rho}=\sum_\alpha p_\alpha\ket{\psi_\alpha}\bra{\psi_\alpha}.
\end{equation}
混态的典型例子是自旋 1/2 的无偏极化束流,1/2 的概率分别处于自旋向上与自旋向下态。总之,密度算符最重要的作用还是描述波函数描述失效的统计系综,体系按一定的概率分布在波函数上。对核多体问题,密度算符主要是用来算可观测量的,在构建自然基时也有用到。

这里其实有个问题,为什么密度算符要用态的外积来定义?这其实是公理层面的问题,并不能说出原因,问这个问题就相当于问为什么要用波函数来描述体系一样。\textbf{波函数描述与密度算符描述是平行的},用密度算符也可以完整给出量子力学的四个基本假设。

\subsubsection{量子力学的四个基本假设}
量子力学的基本假设是什么在不同文献中都不一样,也没有权威的总结,因此这里简单地引用较新研究 \cite{2021carcassi} 的总结,列出以下四个假设(当然 \cite{2021carcassi} 的主要工作反倒是在说四个基本假设并不独立,但这种试图将物理公理化的探讨正确性存疑):
\begin{enumerate}
  \item State Postulate:系统的纯态由复希尔伯特空间中对应于一组非零向量 \(\ket{\psi}\) 的射线 \(\underline{\psi}\)描述,且系统的可观测属性由作用在该空间上的自伴算符(厄米算符)描述。
  \item Measurement Postulate:给定系统处于态 \(\underline{\psi}\),测量属性 \(X\),\(X\) 由具有谱分解的算符
  \begin{equation*}
    \sum_{x,i}\frac{x\op{x_i}{x_i}}{\ip{x_i}{x_i}}
  \end{equation*}
  描述。测量得到值 \(x\) 的概率为
  \begin{equation*}
    P(x|\psi)=\sum_i\frac{|\ip{\psi}{x_i}|^2}{\ip{x_i}{x_i}\ip{\psi}{\psi}}.
  \end{equation*}
  这个假设即为 Born 规则。
  \item Composite System Postulate:复合系统的状态空间由组分系统空间的张量积给出。
  \item Time Evolution Postulate:孤立系统的时间演化由作用于代表系统状态的向量上的幺正算符描述,或者等价地,由薛定谔方程描述。
\end{enumerate}

这里有些描述需要注意。首先是射线,这与通常说的只有一端可延长的射线不是一个概念,而是指的一个一维的复平面。态的模长与方向不管怎么改变都代表同一个态。另外,有了测量假设,就不再把算符的平均值,本征值之类的说法再单列出来了。

显然,如果用密度算符来描述纯态,只需限制密度算符是厄米的,非负定的且 \(\tr\hat{\rho}=1\) 即可,迹为 1 的限制与密度算符的定义已经消除了模长与复相位的影响。而测量得到 \(x\) 的概率可以用 \(\tr(\hat{\rho} P_x)\) 表示。

\subsubsection{约化密度矩阵}
在确认了密度算符确实是与波函数平行的体系基本量,因此不再纠结 \ref{eqs:density-matrix-pure} 的定义是怎么来的之后,开始实际用密度算符处理问题。在核多体问题中用不到混态。组态相互作用理论中 \(\ket{\Psi}\) 是一系列组态的叠加,令组态空间维数为 \(D\),因此密度矩阵变成了 \(D\) 维方阵,与直接用波函数处理问题的难度相同。这里要注意虽然纯态密度矩阵看上去有 \(D^2\) 个元素,但自由度并不比波函数多,因为纯态密度矩阵有 \(\hat{\rho}^2=\hat{\rho}\) 等强约束;也可以在数学上证明,波函数所在的空间与密度矩阵所在的空间是双射,因此是同构的(表述比较随意,意思是这个意思)。

在实际计算中有用的密度矩阵指的是一体与两体的约化密度矩阵(RDM),一体与两体算符的期望值(与跃迁矩阵元)都可以用两体约化密度矩阵计算。而约化密度矩阵与完整的密度矩阵的关系是对 \(N\) 体系统中很多用不到的自由度求和掉(求偏迹)。下面从单体密度的二次量子化出发,导出单体约化密度矩阵。

单体密度算符定义为
\begin{equation}
  \hat{\rho}(\bm{r})=\sum_{i=1}^N\delta(\bm{r}-\bm{r}_i),
\end{equation}
这里一定要注意两个密度的概念不同,要区分作为特定“态”性质的密度算符,以及这里定义的作为可观测量的密度算符。这个 \(\hat{\rho}(\bm{r})\) 代表在 \(\bm{r}\) 处的密度这一个可观测量,并不与任何态的信息挂钩,如果想知道态 \(\ket{\Psi}\) 在 \(\bm{r}\) 处的密度是多少,需要计算 \(\expval{\hat{\rho}(\bm{r})}{\Psi}\)。
而 \ref{eqs:density-matrix-pure} 定义的则是态 \(\ket{\Psi}\) 的基本性质,更准确的说就是一个态,与波函数完全平行。为了区分,记为 \(\hat{\rho}_\Psi\),在这个定义中并未指定任何表象,因此指定坐标表象之后,有 
\begin{equation}
  \expval{\hat{\rho}(\bm{r})}{\Psi}=\expval{\hat{\rho}_\Psi}{\bm{r}}.
\end{equation}

接下来对坐标空间的单体密度算符进行二次量子化,取基矢为 \(\{\phi_p(\bm{r})\}\),因此执行标准的二次量子化流程,计算 
\begin{equation}
  \rho_{pq}=\int \dd\bm{r}'\phi_p^*(\bm{r}')\delta(\bm{r}-\bm{r'})\phi_q(\bm{r}')=\phi_p^*(\bm{r})\phi_q(\bm{r}),
\end{equation}
积分中 \(\bm{r}'\) 是某个粒子的坐标,而 \(\bm{r}\) 是选定的位置坐标,是与密度算符本身绑定的,每个 \(\bm{r}\) 都有一个这样的密度算符。
可以看出二次量子化后是与 \(\bm{r}\) 相关的。这也是自然的。与常见的哈密顿量二次量子化区别是哈密顿量中的坐标都是每个粒子的坐标,不含额外的外部坐标,整个体系就这么一个哈密顿量,不存在分布的概念。而如果考虑哈密顿量密度或拉格朗日量密度的话,就是一个场了,在空间的不同位置是有分布的。

总之 \(\rho(\bm{r})\) 二次量子化后为
\begin{equation}
  \hat{\rho}(\bm{r})=\sum_{pq}\phi_p^*(\bm{r})\phi_q(\bm{r})a_p^\dag a_q,
\end{equation}
因此可计算 
\begin{equation}
  \expval{\hat{\rho}(\bm{r})}{\Psi}=\sum_{pq}\phi_p^*(\bm{r})\phi_q(\bm{r})\mel{\Psi}{a_p^\dag a_q}{\Psi}.
\end{equation}

直接将 \(\hat{\rho}_\Psi\) 写到坐标表象下自然可以,但因为算符 \(\hat{\rho}(\bm{r})\) 二次量子化后,发现只是一个单体算符,因此期望值 \(\expval{\hat{\rho}(\bm{r})}{\Psi}\) 只是态 \(\ket{\Psi}\) 的单体密度,也就是 \(\hat{\rho}_\Psi\) 求偏迹后约化到单体空间的形式。这里进行算符的分解
\begin{equation}
  \hat{A}=\sum_{ij}A_{ij}\op{i}{j},
\end{equation}
这个分解可以在任何空间中进行,甚至可以直接在 Fock 空间中进行,只是在 Fock 空间基矢 \(\ket{i}\) 被产生湮灭算符编码,没法明显地写出某个态矢作为基矢。
在单体空间对密度算符进行分解,也就是把基矢 \(\ket{i}\) 只取为单体空间的基矢,因此 
\begin{equation}
  \expval{\hat{\rho}_\Psi}{\bm{r}}=\sum_{pq}\gamma_{pq}\phi_p(\bm{r})\phi_q^*(\bm{r}),
\end{equation}
这样就得到了单体约化密度矩阵(1RDM)为 
\begin{equation}
  \gamma_{pq}=\mel{\Psi}{a_q^\dag a_p}{\Psi}.
\end{equation}

以上推导绕了非常大的弯子,但是确实好理解,也能直观地看到为什么 \(\gamma_{pq}\) 的下标顺序要与产生湮灭算符的顺序相反,而不是单纯地为了凑迹的定义。更严格也最直接的定义方式是直接对 \ref{eqs:density-matrix-pure} 求偏迹,感觉有些复杂,以后有时间再尝试补充。

在实际使用约化密度矩阵进行计算时,指标的顺序不会这么严格地倒着取,而是与哈密顿量的 Fock 空间矩阵元保持一致,也就是 
\begin{equation}
  \gamma_{pq}=\mel{\Psi}{a_p^\dag a_q}{\Psi},
\end{equation}
同理两体约化密度矩阵(2RDM)取为
\begin{equation}
  \Gamma_{pqrs}=\mel{\Psi}{a_p^\dag a_q^\dag a_s a_r}{\Psi}.
\end{equation}
这样算符
\begin{equation}
  \hat{O}=\sum_{pq}O^{(1)}_{pq}a_p^\dag a_q+\frac14\sum_{pqrs}O^{(2)}_{pqrs}a_p^\dag a_q^\dag a_s a_r
\end{equation}
的期望值应该写为
\begin{equation}
  \expval{O}=\sum_{pq}O^{(1)}_{pq}\gamma_{pq}+\frac14\sum_{pqrs}O^{(2)}_{pqrs}\Gamma_{pqrs}.
\end{equation}
在 \cite{2014overy,2017blunt} 等量子化学的相关文献中都是这么写的。我猜测在核物理平均场的相关教材中用严格的反过来的写法是因为平均场处理的都是单体密度,不处理两体密度,两个指标反一下很好理解,可以接受,而四个指标反的话看着就非常难受。在接下来的推导中统一使用这种对齐的记号,求期望值的时候需要看清楚下标顺序。

另外,当左右两侧的态不同时,也可以定义跃迁密度矩阵。
初态与末态分别记为 \(\ket{\Psi_i}\) 与 \(\ket{\Psi_f}\),最标准,与 \cite{brown,2007suhonen} 相一致的记号就是把末态 \(\ket{\Psi_f}\) 放在左侧,初态 \(\ket{\Psi_i}\) 放在右侧。因此可以计算跃迁矩阵元
\begin{equation}
  O_{fi}=\langle \Psi_f|\hat{O}|\Psi_i\rangle,
\end{equation}
可以用密度矩阵计算 \(O_{fi}\),利用迹的性质可写为
\begin{equation}
  O_{fi}=\langle \Psi_f|\hat{O}|\Psi_i\rangle = \Trace(|\Psi_i\rangle\langle\Psi_f|\hat{O})=\Trace(\rho\hat{O}),
\end{equation}
因此跃迁密度矩阵定义为
\begin{equation}
  \hat{\rho} = |\Psi_i\rangle\langle\Psi_f|,
\end{equation}
这个顺序反而成了初态在左侧,末态在右侧,所以在 \cite{2017blunt} 中将密度矩阵的上标与初末态计算矩阵元的顺序反过来写,即为
\begin{equation}
  \hat{\rho}^{if} = |\Psi_i\rangle\langle\Psi_f|.
\end{equation}
相应的 1RDM 与 2RDM 的上标也对应是 \(if\) 顺序。

\subsection{约化跃迁密度}
约化这个词在不同语境下的含义不同。在 Wigner-Eckart 定理处,约化指的是移除 \(m\) 量子数依赖;而约化密度矩阵的约化指的是对多体系统中不需要的粒子自由度求和掉。不过在实际多体计算中,密度矩阵就是约化到单体与两体的。

核物理还有好的角动量量子数,因此算符也都是给出的约化矩阵元,因此 \cite{brown} 用 \(m\)-scheme 的波函数计算了跃迁密度,包括单体跃迁密度(OBTD)与两体跃迁密度(TBTD),这些跃迁密度给的都是约化后的双线矩阵元。所以原则上这些密度矩阵实际上经历过两次约化,不过单体与两体的这个称呼,已经包含了一层约化的含义,因此也有将 OBTD 与 TBTD 称为约化的单体与两体密度矩阵的,比如 BIGSTICK 的手册与 \cite{2007suhonen},此时约化指的就是变为双线矩阵元。不过在 \cite{2007suhonen} 中将 \(\langle \Psi_f|\hat{O}^\lambda_\mu|\Psi_i\rangle=\langle \xi_fJ_fM_f|\hat{O}^\lambda_\mu|\xi_iJ_iM_i\rangle\) 称为单体跃迁密度了。这些称呼有些混乱,不过都是细节,知道接下来推导的 OBTD 与 TBTD 都是约化后的双线矩阵元就够了。

\subsection{单体跃迁密度}
单体算符在 \(m\)-scheme 下写为二次量子化的形式为
\begin{equation}
  \hat{O}^\lambda_\mu=\sum_{\alpha\beta}\langle \alpha|\hat{O}^\lambda_\mu|\beta\rangle a_\alpha^\dag a_\beta,\label{eqs:one-body-op}
\end{equation}
这里的 \(\alpha\) 与 \(\beta\) 是 \(m\)-scheme 下单粒子态的标号,包含所有量子数。也正因如此才能写出产生湮灭算符,\(j\)-scheme 的单粒子态自然是写不成产生湮灭算符的。

在此不用 \cite{brown} 的引入球张量算符张量积的算法,而是最直接的在 \(m\)-scheme 下计算算符的跃迁矩阵元,然后寻找能够将算符的约化矩阵元与波函数部分分离的公式形式,由此直接定义 OBTD。这么做在概念上更加直观且容易理解,即使引入了张量积,最终还是需要回到 \(m\)-scheme 计算。看上去引入张量积只是为了将 OBTD 定义得更加有物理意义。
直接计算
\begin{equation}
  \mel{\Psi_f}{\hat{O}^\lambda_\mu}{\Psi_i}=\sum_{\alpha\beta}\langle \alpha|\hat{O}^\lambda_\mu|\beta\rangle \mel{\Psi_f}{a_\alpha^\dag a_\beta}{\Psi_i},
\end{equation}
重点是把算符的两个单线矩阵元全部约化为双线,右侧变为
\begin{align}
  \sum_{\alpha\beta}\langle\alpha|\hat{O}^\lambda_\mu|\beta\rangle\mel{\Psi_f}{a_\alpha^\dag a_\beta}{\Psi_i}
  &=\sum_{k_\alpha k_\beta}\sum_{m_\alpha m_\beta} \hat{j_\alpha}^{-1}C_{j_\beta m_\beta \lambda \mu}^{j_\alpha m_\alpha} \langle k_\alpha||\hat{O}^\lambda||k_\beta\rangle \mel{\Psi_f}{a_\alpha^\dag a_\beta}{\Psi_i}\notag\\
  &=\sum_{k_\alpha k_\beta}\langle k_\alpha||\hat{O}^\lambda||k_\beta\rangle\sum_{m_\alpha m_\beta}\hat{\lambda}^{-1}(-1)^{j_\beta - m_\beta}C_{j_\alpha m_\alpha j_\beta -m_\beta}^{\lambda \mu}\mel{\Psi_f}{a_\alpha^\dag a_\beta}{\Psi_i},
\end{align}
这样已经把算符的矩阵元与波函数部分分离开了。还需要再对左侧约化,为了与 KSHELL 代码的公式保持一致,使用 \ref{eqs:w-e-cg1},
\begin{equation}
  \mel{\Psi_f}{\hat{O}^\lambda_\mu}{\Psi_i}=\hat{J_f}^{-1}C_{J_i M_i \lambda \mu}^{J_f M_f}
  \langle \Psi_f||\hat{O}^\lambda||\Psi_i\rangle=(-1)^{J_i-M_i}\hat{\lambda}^{-1}C_{J_iM_iJ_f-M_f}^{\lambda -\mu}\langle \Psi_f||\hat{O}^\lambda||\Psi_i\rangle,
\end{equation}
虽然看上去很奇怪,用到了 \(-\mu\),不过 KSHELL 写的全是初态的 \(M_i\)(代码中记为右态的 \(M_R\))减去末态的 \(M_f\)(代码中为 \(M_L\)),没显式用到这里定义的 \(\mu=M_f-M_i\)。用 CG 系数而不是 \(3j\) 符号还有一个好处是 \(\hat{\lambda}\) 因子被移除了,不用担心 OBTD 定义的问题。
因此得到算符的约化跃迁矩阵元
\begin{align}
  \langle \Psi_f||\hat{O}^\lambda||\Psi_i\rangle &=\frac{(-1)^{J_i-M_i}}{C_{J_iM_iJ_f-M_f}^{\lambda -\mu}}\sum_{k_\alpha k_\beta}\langle k_\alpha||\hat{O}^\lambda||k_\beta\rangle
  \sum_{m_\alpha m_\beta}(-1)^{j_\beta - m_\beta}C_{j_\alpha m_\alpha j_\beta -m_\beta}^{\lambda \mu}\mel{\Psi_f}{a^\dag_\alpha a_\beta}{\Psi_i}\notag\\
  &\equiv\sum_{k_\alpha k_\beta}\langle k_\alpha||\hat{O}^\lambda||k_\beta\rangle\;\text{OBTD}(fik_\alpha k_\beta\lambda),\label{eqs:tr-mat-obtd}
\end{align}
因此 OBTD 就可以定义为
\begin{equation}
  \text{OBTD}(fik_\alpha k_\beta\lambda)=\frac{(-1)^{J_i-M_i}}{C_{J_iM_iJ_f-M_f}^{\lambda -\mu}}\sum_{m_\alpha m_\beta}(-1)^{j_\beta - m_\beta}C_{j_\alpha m_\alpha j_\beta -m_\beta}^{\lambda \mu}\mel{\Psi_f}{a^\dag_\alpha a_\beta}{\Psi_i}.
\end{equation}
这就是完全的 \(j\)-scheme 下的表达式了,算符的全部物理信息都由其约化矩阵元给出,而波函数的信息全部被分离在 OBTD 中。后续\ref{sec:obs-operator} 会将核物理计算常用的各种算符的约化矩阵元给出。
因为波函数在 \(m\)-scheme 下,所以计算 OBTD 也必须先在 \(m\)-scheme 下计算出密度矩阵 \(\gamma_{\alpha\beta}^{if}=\mel{\Psi_f}{a_\alpha^\dag a_\beta}{\Psi_i}\)。

从计算量角度来看,计算 OBTD 时需要对中间夹的 \(j\)-scheme 轨道 \(k_\alpha\) 与 \(k_\beta\) 计算 CG 系数 \(C_{j_\alpha m_\alpha j_\beta -m_\beta}^{\lambda \mu}\),才能把 \(m\)-scheme 的密度矩阵转为 OBTD;但因为算符本身都是直接构建约化矩阵元,如果想用初末态波函数在 \(m\)-scheme 下夹算符还是需要用 CG 系数 \(C_{j_\beta m_\beta\lambda\mu}^{j_\alpha m_\alpha}\) 才能得到单线矩阵元,因此两种方式计算量是一致的。区别只在于保存层面,如果想确实在内存中构建出密度矩阵的话,就只能存 OBTD 而不能是 \(m\)-scheme 的 \(\gamma_{\alpha\beta}^{if}\)。

注意 OBTD 的基矢是 \(j\)-scheme 的单粒子态 \(k\),任两个单粒子态都可以计算,但对给定的算符而言,\(k_\alpha\) 与 \(k_\beta\) 不满足算符本身的对称性时,算符值本身就是 0,自然也没必要计算这个矩阵元。对称性是宇称与同位旋,OBTD 的矩阵元左右的单粒子态必须与初末态波函数的宇称与同位旋对应才能得到非 0 的值。利用这些对称性可以稀疏存储 OBTD。

OBTD 也是依赖于秩 \(\lambda\) 的,实际计算中是先给定跃迁的初末态,通过初末态角动量的耦合确定能发生跃迁的算符秩的范围 \(|J_f-J_i|\le\lambda\le J_f+J_i\),然后计算每个 \(\lambda\) 对应的 OBTD。以 KSHELL 用 USDB 计算 \ce{^24Mg} 的 OBTD 为例,如果初末态都是 \(0^+\),即使是 \(0d_{5/2}\) 轨道之间的 OBTD,也只需要计算 \(\lambda=0\)。

\subsubsection{标量算符}



\subsection{两体跃迁密度}
两体算符在 \(m\)-scheme 下写为二次量子化的形式为
\begin{equation}
  \hat{T}^\lambda_\mu=\frac{1}{4}\sum_{\alpha\beta\gamma\delta}\langle \alpha\beta|\hat{T}^\lambda_\mu|\gamma\delta\rangle a_\alpha^\dag a_\beta^\dag a_\delta a_\gamma,\label{eqs:two-body-op}
\end{equation}
\(m\)-scheme 处于非耦合表象,因此需要对两体算符进行两步处理,先用 \ref{eqs:j-mel-to-m} 将非耦合表象的单线矩阵元用耦合表象下的矩阵元表示出来,此时初态与末态才有好角动量,才能用 \ref{eqs:wigner-eckart} 将单线矩阵元转为约化矩阵元,也即
\begin{align}
  \hat{T}^\lambda_\mu ={}& \frac14\sum_{k_\alpha k_\beta k_\gamma k_\delta}\sum_{m_\alpha m_\beta m_\gamma m_\delta}\mel{k_\alpha m_\alpha k_\beta m_\beta}{\hat{T}^\lambda_\mu}{k_\gamma m_\gamma k_\delta m_\delta}a^\dag_{\alpha} a^\dag_{\beta}a_{\delta}a_{\gamma}\notag\\
  ={}&\frac14\sum_{k_\alpha k_\beta k_\gamma k_\delta}\sum_{JMJ'M'}\frac1{N_{k_\alpha k_\beta}N_{k_\gamma k_\delta}}\mel{k_\alpha k_\beta JM}{\hat{T}^\lambda_\mu}{k_\gamma k_\delta J'M'}\notag\\
  &\times\sum_{m_\alpha m_\beta m_\gamma m_\delta}C_{j_\alpha m_\alpha j_\beta m_\beta}^{JM}C_{j_\gamma m_\gamma j_\delta m_\delta}^{J'M'}a^\dag_{\alpha} a^\dag_{\beta} a_{\delta} a_{\gamma},
\end{align}
计算单线矩阵元
\begin{align}
  \mel{\Psi_f}{\hat{T}^\lambda_\mu}{\Psi_i} ={}&\frac14\sum_{k_\alpha k_\beta k_\gamma k_\delta}\sum_{JMJ'M'}\frac1{N_{k_\alpha k_\beta}N_{k_\gamma k_\delta}}\mel{k_\alpha k_\beta JM}{\hat{T}^\lambda_\mu}{k_\gamma k_\delta J'M'}\notag\\
  &\times\sum_{m_\alpha m_\beta m_\gamma m_\delta}C_{j_\alpha m_\alpha j_\beta m_\beta}^{JM}C_{j_\gamma m_\gamma j_\delta m_\delta}^{J'M'}\mel{\Psi_f}{a^\dag_{\alpha} a^\dag_{\beta} a_{\delta} a_{\gamma}}{\Psi_i}\notag\\
  ={}&\frac14\sum_{k_\alpha k_\beta k_\gamma k_\delta}\sum_{JJ'}\frac1{N_{k_\alpha k_\beta}N_{k_\gamma k_\delta}}\langle k_\alpha k_\beta J||\hat{T}^\lambda||k_\gamma k_\delta J'\rangle\notag\\
  &\times\sum_{MM'}\hat{\lambda}^{-1}(-1)^{J'-M'}C_{J M J' -M'}^{\lambda \mu}\sum_{m_\alpha m_\beta m_\gamma m_\delta}C_{j_\alpha m_\alpha j_\beta m_\beta}^{J M}C_{j_\gamma m_\gamma j_\delta m_\delta}^{J' M'}\mel{\Psi_f}{a^\dag_{\alpha} a^\dag_{\beta} a_{\delta} a_{\gamma}}{\Psi_i}, 
\end{align}
等式左侧的约化与单体算符一致,
\begin{equation}
  \mel{\Psi_f}{\hat{T}^\lambda_\mu}{\Psi_i}=\hat{J_f}^{-1}C_{J_i M_i \lambda \mu}^{J_f M_f}
  \langle \Psi_f||\hat{T}^\lambda||\Psi_i\rangle=(-1)^{J_i-M_i}\hat{\lambda}^{-1}C_{J_iM_iJ_f-M_f}^{\lambda -\mu}\langle \Psi_f||\hat{T}^\lambda||\Psi_i\rangle,
\end{equation}
因此得到算符的约化跃迁矩阵元
\begin{align}
  \langle \Psi_f||\hat{T}^\lambda||\Psi_i\rangle ={}&\frac14\frac{(-1)^{J_i-M_i}}{C_{J_iM_iJ_f-M_f}^{\lambda -\mu}}\sum_{k_\alpha k_\beta k_\gamma k_\delta}\sum_{JJ'}\frac1{N_{k_\alpha k_\beta}N_{k_\gamma k_\delta}}\langle k_\alpha k_\beta J||\hat{T}^\lambda||k_\gamma k_\delta J'\rangle\notag\\
  &\times\sum_{MM'}(-1)^{J'-M'}C_{J M J' -M'}^{\lambda \mu}\sum_{m_\alpha m_\beta m_\gamma m_\delta}C_{j_\alpha m_\alpha j_\beta m_\beta}^{J M}C_{j_\gamma m_\gamma j_\delta m_\delta}^{J' M'}\mel{\Psi_f}{a^\dag_{\alpha} a^\dag_{\beta} a_{\delta} a_{\gamma}}{\Psi_i}\notag\\
  \equiv{}&\sum_{k_\alpha\le k_\beta,k_\gamma\le k_\delta}\sum_{JJ'}\;\langle k_\alpha k_\beta J||\hat{T}^\lambda||k_\gamma k_\delta J'\rangle\;\text{TBTD}(fikJJ'\lambda),\label{eqs:tr-mat-tbtd}
\end{align}
TBTD 的定义为 
\begin{align}
  \text{TBTD}(fikJJ'\lambda)={}&\frac{(-1)^{J_i-M_i}}{C_{J_iM_iJ_f-M_f}^{\lambda -\mu}}\sum_{MM'}(-1)^{J'-M'}C_{J M J' -M'}^{\lambda \mu}\notag\\
  &\times\sum_{m_\alpha m_\beta m_\gamma m_\delta}C_{j_\alpha m_\alpha j_\beta m_\beta}^{J M}C_{j_\gamma m_\gamma j_\delta m_\delta}^{J' M'}\mel{\Psi_f}{a^\dag_{\alpha} a^\dag_{\beta} a_{\delta} a_{\gamma}}{\Psi_i},
\end{align}
之前在 \ref{sec:second-quantum} 讨论过二次量子化形式中系数 1/4 的问题。这里要注意已经写到了 \(j\)-scheme 下,因此是允许 \(k_\alpha=k_\beta\) 或 \(k_\gamma=k_\delta\) 的,这就是归一化因子 \(N_{k_\alpha k_\beta}\) 干的事情。当 \(k_\alpha=k_\beta\) 时,\(N_{k_\alpha k_\beta}=1/\sqrt{2}\),因此在 \ref{eqs:tb-op-to-tensor-prod} 中,\(k_\alpha=k_\beta\) 时确实求和了两次,直接将 1/4 改为 \(k_\alpha\le k_\beta\) 是正确的。

对比 OBTD,区别在于 OBTD 只需在 \(m\)-scheme 下计算单体密度矩阵 \(\gamma^{if}_{\alpha\beta}\) 代入即可,而 TBTD 在计算两体密度矩阵 \(\Gamma^{if}_{\alpha\beta\gamma\delta}\) 之后,还需要耦合为角动量为 \(J\) 与 \(J'\) 的两体态。这个耦合是必须的,因为算符作用于好角动量的态之间,算符矩阵元的左态与右态必定有好的角动量量子数,拿着非耦合表象下的两体密度矩阵是什么也算不出的。