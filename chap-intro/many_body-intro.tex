\section{模型、理论与方法}
在具体讨论多体方法之前,首先要说明一下我理解的物理学中“模型”(model)“理论”(theory)“方法”(method 或 approach)这些基本概念。想完全描述一个真实的物理体系是极其复杂的,所以会对物理图像进行一些简化与近似,这样就得到了“模型”,比如原子核的液滴模型,是把原子核看作一团带电液滴。模型与理论不要搞混,模型一定是针对于某一个具体的物理体系而言的,比如原子核结构模型,甚至粒子物理标准模型;而理论是抽象的,可以脱离现实中的对象而存在,或者说描述的对象只存在于概念中,比如质点动力学、量子力学、量子场论。可以认为模型对体系进行了简化,使得理论能够应用于这个体系。

理论如何用于描述该体系的这个过程,可以被认为是“方法”,也就是实操中的具体做法。方法可以认为是绑定了对象的理论。那些写在课本上的,具有极高普适性的东西可称为理论,而实际上,绝大多数研究都是针对对象的,所以用到的“理论”其实都是这里定义的“方法”。因此在很多(针对某个对象进行研究,而不是讲最普适的理论体系的)文献与课本中,理论与方法这两个词一般不进行区分,很多时候是混用的。

这里要着重提一下标准模型。虽然标准模型被视为当代物理学的最伟大成就之一,但它不是理论,只是一个粒子物理中的模型,描述的对象是基本粒子。标准模型用到了规范场论等一系列的理论,而这些理论描述的基本粒子,也不是真实的粒子(实际上没人知道真实的粒子应该是什么样子),而是处于 SU(2) 群表示的轻子,SU(3) 群表示的夸克,等等。把粒子写为群表示其实就可以理解为某种模型,只有这么简化,才可以用这些群的规范理论去描述这些粒子。另外标准模型也有多个参数,这些参数并非来自于更底层的理论,而是通过实验得到。这表明标准模型并非所寻找的终极理论,而可能的终极理论比如大统一理论 GUT,就用了“理论”这个词,表明它不局限于特定的对象。

介绍标准模型的目的,是为了给核物理吃下一颗定心丸:“模型”并不是一个带有贬义色彩的术语,连标准模型都只是一个“模型”,那核物理也通过建立一系列原子核模型的方式,又有何不可?因此,严格来说,整个(针对对象的)物理学都是模型,因为一旦抽象的“理论”被用于具体的物理体系,就势必要对这个体系进行一定程度的假设与近似,区别只在于近似的多少,舍弃了多少细节。以原子核模型为例进行说明,液滴模型、费米气体模型是很大程度的近似,舍弃了很多细节,而“原子核可以用核子与介子作为自由度进行研究”,这也是模型,只是这个近似不算很大,绝大多数人应该都认同。即使秉承还原论的观点,把原子核彻底还原到夸克、胶子层面进行研究,也躲避不了标准模型也只是一个模型的这个事实。

但是,核理论研究者也不能因此就整天对全世界承认“我们的研究只是模型而已”。尽管我们自己内心知道整个物理学都很难脱离模型,但如果近似的内容只是一些人尽皆知的,被绝大多数人认为是没问题的东西,就很少有人用最严格的判断标准,强调其模型属性,比如核物理的手征有效场论,以及自由度选为核子,这些严格意义上都是模型,但并不会有人特意指出这一点。相反,液滴模型、费米气体模型、壳模型、集体模型、平均场模型等,从命名上就被称为模型,这是因为相比绝大多数人公认的原子核由核子构成的物理图像,这些模型的近似程度过大了——液滴模型将原子核看作带电液滴,费米气体模型看作无相互作用费米气体,壳模型与平均场模型看作在其他核子形成的平均场中的独立粒子。正因如此,这些与公认图像近似程度较大的模型,才是常用的“模型”的含义。

也就是说,存在一个对体系的简化、近似程度的分界线,如果近似程度低于此就不会被强调是模型,反之则会被认为是模型。我给出的这条分界线的标准是“物质世界的层次划分”。在对某个体系进行研究时,总会选择适合这个体系的自由度,比如核物理选为核子,化学可能选分子、原子或电子。这种自由度的选择本质上是严格意义上的模型,因为并不能保证更底层的自由度对体系无影响。然而,选取了恰当的自由度之后,如果使用的理论与方法不需要对体系的物理图像进行进一步的简化,就能完成求解,\underline{在通常语境中},这样的方法就不会被强调为模型。但如果在求解之时进一步对物理图像做了简化,就是通常语境的模型。总的来说,模型是一个相对的概念,如果比“绝大多数人都认为不是模型的模型”更是模型,就是绝大多数人都认为的、通常语境的“模型”。

\section{原子核的理论计算}
\subsection{\textit{Ab initio} 计算}
以上讨论放到核物理中,选取核子为自由度,在这个“绝大多数人都认为不是模型的模型”之下,所需的只有核力与多体方法。如果使用最先进的手征有效场论核力为代表的现实核力,并用严格的量子多体方法对原子核进行理论计算,就没有进一步的简化,这样的做法就可以不叫模型。正巧,这也是 \textit{ab initio} 计算的定义,因此在通常的语境中可以说,\textit{ab initio} 计算不是模型,模型不是 \textit{ab initio} 的。而常用的 \textit{ab initio} 方法也有成为模型的,比如无核芯壳模型,只是在历史沿革上借助了壳模型的概念而起的名字而已,并不是说它是一种模型。

对于现实核力是不是模型的讨论并没有这么简单,虽然 \textit{ab initio} 计算所用的多体方法已经脱离了模型的范畴,但那些唯象的现实核力或者介子交换的现实核力应该做了一些模型的假设。由于多体方法才是本文的重点,这里就仅仅下几个暂时的定论:

1. 最先进的手征有效场论核力可以不认为是模型,其他的现实核力可能是模型,但因为我不了解所以不妄下定论。

2. 尽管如此,也不能因为某些现实核力是模型,就连带地认为整个 \textit{ab initio} 计算都是模型。因为从物质世界层次划分的角度来说,即使核子间的相互作用是通过模型的方式获得的,进一步的计算也是在核子层面进行,这符合对原子核层次划分的通常认识,所以现实核力是模型并不会导致 \textit{ab initio} 计算是模型。

已经说明了 \textit{ab initio} 计算不是模型,那 \textit{ab initio} 是理论吗?答案也是否定的。注意在这里的用词全是 \textit{ab initio} 计算或 \textit{ab initio} 方法。不能称为模型是因为 \textit{ab initio} 的性质与定义决定了它不能对原子核进行模型层面的简化,只能进行计算层面的截断;而不能称为理论则是因为 \textit{ab initio} 没有提出新的理论,而是使用已经存在的思想与理论对原子核进行计算,理论就无所谓 \textit{ab initio} 与否了。

虽然在历史上 \textit{ab initio} 计算是最近才兴起的,但仍然将其放在这里的第一位,是因为概念上 \textit{ab initio} 方法比所有模型思想都更直接,关于 \textit{ab initio} 计算用到的多体方法是本章以及本书的重点,在这里就不展开讨论了。由于历史上对核力的性质缺乏了解,多体方法发展不完善以及计算能力的限制,当时无法实现 \textit{ab initio} 计算,因此需要对核力或多体方法进行进一步的简化,建立非 \textit{ab initio} 的原子核模型,以实现对原子核的理论计算。

\subsection{宏观模型}
如果对原子核的物理图像进行了大幅度的简化,以至于把自由度抬升到了高于核子的层次,就会得到原子核的宏观模型。比如液滴模型(其实没有自由度,也没有动力学),集体模型(自由度是液滴的转动与振动),费米气体模型(也是一团没有自由度的费米气体)。关于此处提到的宏观与微观模型,胡济民对微观理论有这样的定义,“从包含核子-核子相互作用的核多体哈密顿量的某种近似解出发的理论,都可以认为是微观理论”,而无论这个相互作用是现实的还是有效的。因此以上的模型在胡济民从相互作用出发的定义中,也属于宏观模型。

但有一个比较反直觉的例外,是建立在费米气体模型上的 Thomas Fermi 近似,其物理图像是费米气体,但引入了核子的有效势,以此建立密度泛函。看上去应该是微观模型,但胡济民称其为“有一定微观基础的宏观模型”。这可能是因为微观模型是“从包含核子-核子相互作用的核多体哈密顿量的某种近似解出发的”,这个近似解很多时候就是 HF,也就是说 HF 是最简单的微观理论,其他的微观模型应该比 HF 更进一步。但 Thomas Fermi 近似“只是 HF 的很好近似”,这可能是被胡济民称为分类为宏观模型的原因。如果不在意这个例外的话,直接从是否使用核子间的相互作用来区分宏观与微观模型应该就足够了。

\subsection{宏观-微观模型}
宏观-微观模型是在液滴模型的基础上,引入壳模型的壳修正。同时这种做法还去除了壳模型中的 double counting 问题。通过对壳模型平均场的调整,宏观-微观模型描述原子核的能力非常强,但在外推到未知区域时不太可靠。

\subsection{微观模型}
微观模型需要使用核子的相互作用。Greiner 指出,必须使用“模型”这个术语,因为缺乏核力的确切可靠的理论 \cite{book-greiner}。但这个论述只在现实核力出现前才对,正如之前的讨论,\textit{ab initio} 计算不是模型,即使现实核力有一定的模型成分。而微观模型相比 \textit{ab initio} 计算对核力或多体方法都进行了一系列的假设与简化,比如壳模型是独立粒子近似(Greiner 称为唯象单粒子模型),而组态相互作用壳模型的近似则是冻结的核芯以及核子只能在价空间填充;平均场模型也是独立粒子近似。

\subsubsection{密度泛函理论}
核物理中使用的平均场模型与量子化学,凝聚态中常用的密度泛函理论(DFT)有联系,但在核物理中却称为自洽平均场(SCMF)\cite{2003bender}。核物理文献会把平均场求解的相互作用的参数称为一套“密度泛函”,比如 PC-PK1 密度泛函,而这套密度泛函可以用很多种理论方法去求解,比如 RMF,RCHB,DRHBc 等,求解的都是同一套密度泛函,不同的是求解的过程中考虑的效应,但却不会把这套理论称为 DFT。

DFT 的基本思想是,对于 $N$ 体系统,其波函数需要 $3N$ 个坐标,求解是极其复杂的,但可以用波函数计算出体系的密度,密度只有 3 个坐标,大大简化了多体系统。在量子化学中,把从波函数出发描述体系的方法称为波函数理论(WFT),与 DFT 并列。

处于外场中的体系能量可以写为密度的泛函,这由 Hohenberg-Kohn 定理保证,这里不详细描述 H-K 定理的具体内容与证明,只阐述其作用。但 H-K 定理只能形式上证明 DFT 的正确性,但并不知道怎么写出这个能量作为密度的泛函。此时就需要用到 Kohn 与 Sham 提出的 ansatz(拟设),将相互作用的多体系统映射到无相互作用的,处于外场 $v_\mathrm{KS}(\boldsymbol{r})$ 中的体系,因此体系只需要在单粒子层面上求解就可以了。这正是核物理中的 DFT 被称为自洽平均场的原因。

但独立粒子近似的简便性总有其代价,DFT 的主要工作是得到 $v_\mathrm{KS}(\boldsymbol{r})$,这个等效外场也是密度的泛函,因此所有问题都转变为能量密度泛函的形式如何,其形式越准确,DFT 的效果就越好。在量子化学中,由于电子的相互作用相对简单,只有库仑相互作用,因此不唯象地得出能量密度泛函是可能的,也发展了分层次逐阶提升泛函所包含关联的一系列方法,关联由少到多有 LDA(局域密度近似),GGA(广义梯度近似),mGGA,hyper-GGA 等等,这就是 John Perdew 提出的 Jacob's ladder。这种思想与核物理的 \textit{ab initio} 计算有着异曲同工之处,从粒子的真实相互作用出发,在密度泛函中逐阶增加关联,虽然 KS 拟设不一定成立,但关联有大部分是能够被包含进去的。所以在量子化学中 DFT 被称为是 \textit{ab initio} 计算是有道理的。

然而核物理中的 DFT 面临核力形式很复杂的挑战,难以从核力出发直接得出密度泛函的形式,因此建立了一系列相对论能量密度泛函,根据场论的基本原理,写出核子、介子相互作用的拉氏量 $\mathcal{L}$ 可能包含的项,并拟合这些项的参数,然后求解单粒子在这样的场中的运动。选择以相对论的形式构建能量密度泛函是可以理解的,毕竟核力的形式未知,非相对论的密度泛函要做出的假设更多,从最底层的量子场论出发构建相对论的拉氏量肯定是不会错的选择。因此拉氏量的公式形式是 \textit{ab initio} 的 \cite{2003bender},但需要拟合系数。在拟合中关联也被吸收进拉氏量中了,所以现在常用的平均场虽然是建立在独立粒子近似的假设上,但也能很好地描述核版图中大量核的性质。不过核物理的平均场模型仍然不是 \textit{ab initio} 有两个原因,一是因为密度泛函拟合的范围太大了,虽然参数数量与手征有效场论类似,都是十几个,但手征有效场论只拟合散射相移与少体可观测量,与原子核的层次划分保持一致,而密度泛函拟合了大量多体系统的可观测量,破坏了核子为自由度的这种层次划分;二是因为独立粒子近似可能不成立,也缺少系统性包含超出平均场关联的改进方案。

\subsubsection{TDA 与 RPA}
这两种方法在 HF 的基础之上考虑了 $1p1h$ 关联以及组态混合,其实没有什么物理图像上的近似,只是对激发态做了截断与限制,核力也完全可以使用现实核力,没有限制。从这个角度来看 TDA 与 RPA 不是模型。

\section{\textit{Ab initio} 计算用到的多体方法}
前面已经讨论了 \textit{ab initio} 计算与原子核的多种模型,并澄清了 \textit{ab initio} 计算不是模型。但什么是 \textit{ab initio} 计算呢?现实核力好理解,但什么是严格的、误差可估计、可逐级改善的多体方法呢?使用什么样的多体方法才足够 \textit{ab initio},很多人都在争论,也没有一个确定的结果。因此这里也不对这个问题多加讨论,只是给一个广泛的说明:只要不对多体系统额外做更多的假设,而是去努力包含关联,这种计算就可以被认为是 \textit{ab initio} 的。也就是说 \textit{ab initio} 其实是一个相对的概念,很难说哪种做法 \textit{ab initio},哪种做法不 \textit{ab initio}。RPA 肯定比 HF 更 \textit{ab initio},而 IMSRG(2),CCSD 又比 RPA 更 \textit{ab initio}。

目前常用于 \textit{ab initio} 计算的方法在综述 \cite{2020hergert} 中有总结,在我们自己拿不准定义的情况下,相信主流的定义总是一个好办法。用于 \textit{ab initio} 的多体方法实际上都是之前提到过的 WFT,处理的都是波函数。